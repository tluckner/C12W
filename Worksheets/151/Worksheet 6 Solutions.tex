\documentclass[10pt]{article}

\usepackage{enumerate}
\usepackage{amsmath}
\usepackage{amssymb}
\usepackage{amsthm}
\usepackage{array}
\usepackage[all]{xy}
\usepackage{fancyhdr}
\usepackage{euscript}
\usepackage{graphics}
\usepackage{cancel}
\usepackage{fancybox}
\usepackage{tikz}
\usepackage{tikz-3dplot}
\usepackage{pgf}
\usepackage{pgfplots}
\usepackage[all]{xy}
\usepackage{graphicx}
\pgfplotsset{compat=1.14}

\usepackage{pstricks}
\usepackage{pst-plot}

\usepackage{setspace}
\onehalfspacing

\setlength{\oddsidemargin}{.5in}
\setlength{\evensidemargin}{.5in}
\setlength{\textwidth}{6.in}
\setlength{\topmargin}{0in}
\setlength{\headsep}{.20in}
\setlength{\textheight}{8.5in}


\pdfpagewidth 8.5in
 \pdfpageheight 11in


%General
\newcommand{\WW}{\mathbb {W}}
\newcommand{\ZZ}{\mathbb{Z}}
\newcommand{\RR}{\mathbb {R}}
\newcommand{\II}{\mathbb {I}}
\newcommand{\QQ}{\mathbb {Q}}
\newcommand{\CC}{\mathbf C}
\newcommand{\NN}{\mathbb {N}}
\newcommand{\Zn}[1]{\mathbf{Z}/#1\mathbf{Z}}
\newcommand{\Znx}[1]{(\mathbf{Z}/#1\mathbf{Z})^\times}
\newcommand{\X}{\times} 
\newcommand{\set}[2]{\left\{#1 : #2\right\}}          
\newcommand{\sett}[1]{\left\{#1\right\}}                
\newcommand{\nonempty}{\neq\varnothing}
\newcommand{\ds}{\displaystyle}
\newcommand{\abs}[1]{\left| {#1} \right|}
\newcommand{\qedbox}{\rule{2mm}{2mm}}
\renewcommand{\qedsymbol}{\qedbox}											
\newcommand{\aand}{\qquad\hbox{and}\qquad}
\newcommand{\e}{\varepsilon}
\newcommand{\tto}{\rightrightarrows}
\newcommand{\gs}{\geqslant}
\newcommand{\ls}{\leqslant}
\renewcommand{\tilde}{\widetilde}
\renewcommand{\hat}{\widehat}
\newcommand{\norm}[1]{\left\| #1 \right\|}
\newcommand{\md}[3]{#1\equiv#2\;(\mathrm{mod}\;#3)}     
\newcommand{\gen}[1]{\left\langle #1 \right\rangle}
\renewcommand{\Re}{\operatorname{Re}}
\renewcommand{\Im}{\operatorname{Im}}
\newcommand{\zero}{\boldsymbol{0}}

\newcommand{\be}[1]{\textbf{\emph{#1}}}
\newcommand{\hhat}[1]{\hat{\! \hat{#1}}}

\newcommand{\fto}[1]{\xrightarrow{\hspace{4pt} #1 \hspace{4pt}}}
\newcommand{\flto}[1]{\xrightarrow{\quad #1 \quad}}



\newcommand{\dist}{\operatorname{dist}}
\newcommand{\esssup}{\operatorname{ess\:sup}}
\newcommand{\id}{\operatorname{id}}
\newcommand{\card}{\operatorname{card}}

\newcommand{\dmu}{\:\mathrm{d}\mu}
\newcommand{\dm}{\:\mathrm{d}m}
\newcommand{\dx}{\:\mathrm{d}x}
\newcommand{\dt}{\:\mathrm{d}t}
\newcommand{\dz}{\:\mathrm{d}z}
\newcommand{\dtheta}{\:\mathrm{d}\theta}
\newcommand{\dw}{\:\mathrm{d}w}

%Algebra
\newcommand{\Sym}{\operatorname {Sym}}
\newcommand{\Stab}{\operatorname {Stab}}
\newcommand{\M}{\operatorname{M}}
\newcommand{\GL}{\operatorname{GL}}
\newcommand{\PGL}{\operatorname{PGL}}
\newcommand{\SL}{\operatorname{SL}}
\newcommand{\PSL}{\operatorname{PSL}}
\newcommand{\Heis}{\operatorname{Heis}}
\newcommand{\Aff}{\operatorname{Aff}}
\newcommand{\Aut}{\operatorname{Aut}}
\newcommand{\image}{\operatorname{im}}
\newcommand{\Syl}[2]{\operatorname{\emph{Syl}}_{#1}\left(#2\right)}
\newcommand{\Hom}{\operatorname{Hom}}
\newcommand{\Tor}{\operatorname{Tor}}
\newcommand{\Gal}{\operatorname{Gal}}
\newcommand{\ch}{\operatorname{ch}}
\newcommand{\rad}{\operatorname{rad}}
\newcommand{\iso}{\cong}
\newcommand{\normal}{\unlhd}
\newcommand{\semi}{\rtimes}
\newcommand{\Nm}{\operatorname {N}}
\newcommand{\Tr}{\operatorname {Tr}}
\newcommand{\disc}{\operatorname {disc}}








%Euler Script Characters
\newcommand{\esa}{\EuScript{A}}
\newcommand{\esb}{\EuScript{B}}
\newcommand{\esc}{\EuScript{C}}
\newcommand{\esd}{\EuScript{D}}
\newcommand{\ese}{\EuScript{E}}
\newcommand{\esf}{\EuScript{F}}
\newcommand{\esg}{\EuScript{G}}
\newcommand{\esh}{\EuScript{H}}
\newcommand{\esi}{\EuScript{I}}
\newcommand{\esj}{\EuScript{J}}
\newcommand{\esk}{\EuScript{K}}
\newcommand{\esl}{\EuScript{L}}
\newcommand{\esm}{\EuScript{M}}
\newcommand{\esn}{\EuScript{N}}
\newcommand{\eso}{\EuScript{O}}
\newcommand{\esp}{\EuScript{P}}
\newcommand{\esq}{\EuScript{Q}}
\newcommand{\esr}{\EuScript{R}}
\newcommand{\ess}{\EuScript{S}}
\newcommand{\est}{\EuScript{T}}
\newcommand{\esu}{\EuScript{U}}
\newcommand{\esv}{\EuScript{V}}
\newcommand{\esw}{\EuScript{W}}
\newcommand{\esx}{\EuScript{X}}
\newcommand{\esy}{\EuScript{Y}}
\newcommand{\esz}{\EuScript{Z}}

%Calligraphic Characters
\newcommand{\cala}{\mathcal{A}}
\newcommand{\calb}{\mathcal{B}}
\newcommand{\calc}{\mathcal{C}}
\newcommand{\cald}{\mathcal{D}}
\newcommand{\cale}{\mathcal{E}}
\newcommand{\calf}{\mathcal{F}}
\newcommand{\calg}{\mathcal{G}}
\newcommand{\calh}{\mathcal{H}}
\newcommand{\cali}{\mathcal{I}}
\newcommand{\calj}{\mathcal{J}}
\newcommand{\calk}{\mathcal{K}}
\newcommand{\call}{\mathcal{L}}
\newcommand{\calm}{\mathcal{M}}
\newcommand{\caln}{\mathcal{N}}
\newcommand{\calo}{\mathcal{O}}
\newcommand{\calp}{\mathcal{P}}
\newcommand{\calq}{\mathcal{Q}}
\newcommand{\calr}{\mathcal{R}}
\newcommand{\cals}{\mathcal{S}}
\newcommand{\calt}{\mathcal{T}}
\newcommand{\calu}{\mathcal{U}}
\newcommand{\calv}{\mathcal{V}}
\newcommand{\calw}{\mathcal{W}}
\newcommand{\calx}{\mathcal{X}}
\newcommand{\caly}{\mathcal{Y}}
\newcommand{\calz}{\mathcal{Z}}

%Gothic Characters
\newcommand{\fraka}{\mathfrak{a}}
\newcommand{\frakb}{\mathfrak{b}}
\newcommand{\frakc}{\mathfrak{c}}
\newcommand{\frakd}{\mathfrak{d}}
\newcommand{\frake}{\mathfrak{e}}
\newcommand{\frakf}{\mathfrak{f}}
\newcommand{\frakg}{\mathfrak{g}}
\newcommand{\frakh}{\mathfrak{h}}
\newcommand{\fraki}{\mathfrak{i}}
\newcommand{\frakj}{\mathfrak{j}}
\newcommand{\frakk}{\mathfrak{k}}
\newcommand{\frakl}{\mathfrak{l}}
\newcommand{\frakm}{\mathfrak{m}}
\newcommand{\frakn}{\mathfrak{n}}
\newcommand{\frako}{\mathfrak{o}}
\newcommand{\frakp}{\mathfrak{p}}
\newcommand{\frakq}{\mathfrak{q}}
\newcommand{\frakr}{\mathfrak{r}}
\newcommand{\fraks}{\mathfrak{s}}
\newcommand{\frakt}{\mathfrak{t}}
\newcommand{\fraku}{\mathfrak{u}}
\newcommand{\frakv}{\mathfrak{v}}
\newcommand{\frakw}{\mathfrak{w}}
\newcommand{\frakx}{\mathfrak{x}}
\newcommand{\fraky}{\mathfrak{y}}
\newcommand{\frakz}{\mathfrak{z}}

\newcommand{\frakA}{\mathfrak{A}}
\newcommand{\frakB}{\mathfrak{B}}
\newcommand{\frakC}{\mathfrak{C}}
\newcommand{\frakD}{\mathfrak{D}}
\newcommand{\frakE}{\mathfrak{E}}
\newcommand{\frakF}{\mathfrak{F}}
\newcommand{\frakG}{\mathfrak{G}}
\newcommand{\frakH}{\mathfrak{H}}
\newcommand{\frakI}{\mathfrak{I}}
\newcommand{\frakJ}{\mathfrak{J}}
\newcommand{\frakK}{\mathfrak{K}}
\newcommand{\frakL}{\mathfrak{L}}
\newcommand{\frakM}{\mathfrak{M}}
\newcommand{\frakN}{\mathfrak{N}}
\newcommand{\frakO}{\mathfrak{O}}
\newcommand{\frakP}{\mathfrak{P}}
\newcommand{\frakQ}{\mathfrak{Q}}
\newcommand{\frakR}{\mathfrak{R}}
\newcommand{\frakS}{\mathfrak{S}}
\newcommand{\frakT}{\mathfrak{T}}
\newcommand{\frakU}{\mathfrak{U}}
\newcommand{\frakV}{\mathfrak{V}}
\newcommand{\frakW}{\mathfrak{W}}
\newcommand{\frakX}{\mathfrak{X}}
\newcommand{\frakY}{\mathfrak{Y}}
\newcommand{\frakZ}{\mathfrak{Z}}

%Lowercase Bold Letters
\newcommand{\bfa}{\mathbf{a}}
\newcommand{\bfb}{\mathbf{b}}
\newcommand{\bfc}{\mathbf{c}}
\newcommand{\bfd}{\mathbf{d}}
\newcommand{\bfe}{\mathbf{e}}
\newcommand{\bff}{\mathbf{f}}
\newcommand{\bfg}{\mathbf{g}}
\newcommand{\bfh}{\mathbf{h}}
\newcommand{\bfi}{\mathbf{i}}
\newcommand{\bfj}{\mathbf{j}}
\newcommand{\bfk}{\mathbf{k}}
\newcommand{\bfl}{\mathbf{l}}
\newcommand{\bfm}{\mathbf{m}}
\newcommand{\bfn}{\mathbf{n}}
\newcommand{\bfo}{\mathbf{o}}
\newcommand{\bfp}{\mathbf{p}}
\newcommand{\bfq}{\mathbf{q}}
\newcommand{\bfr}{\mathbf{r}}
\newcommand{\bfs}{\mathbf{s}}
\newcommand{\bft}{\mathbf{t}}
\newcommand{\bfu}{\mathbf{u}}
\newcommand{\bfv}{\mathbf{v}}
\newcommand{\bfw}{\mathbf{w}}
\newcommand{\bfx}{\mathbf{x}}
\newcommand{\bfy}{\mathbf{y}}
\newcommand{\bfz}{\mathbf{z}}




%Customized Theorem Environments
\newtheoremstyle%
{custom}%
{}%                         Space above
{}%													Space below
{}%													Body font
{}%                         Indent amount
{}%                         Theorem head font
{.}%                        Punctuation after heading
{ }%                        Space after heading
{\thmname{}%                Additional specifications for theorem head
\thmnumber{}%
\thmnote{\bfseries #3}}%

\newtheoremstyle%
{Theorem}%
{}%
{}%
{\itshape}%
{}%
{}%
{.}%
{ }%
{\thmname{\bfseries #1}%
\thmnumber{\;\bfseries #2}%
\thmnote{\;(\bfseries #3)}}%

%Theorem Environments
\theoremstyle{Theorem}
\newtheorem{theorem}{Theorem}[section]
\newtheorem{cor}{Corollary}[section]
\newtheorem{lemma}{Lemma}[section]
\newtheorem{prop}{Proposition}[section]
\newtheorem*{nonumthm}{Theorem}
\newtheorem*{nonumprop}{Proposition}
\theoremstyle{definition}
\newtheorem{definition}{Definition}[section]
\newtheorem*{answer}{Answer}
\newtheorem*{solution}{Solution}
\newtheorem*{nonumdfn}{Definition}
\newtheorem*{nonumex}{Example}
\newtheorem{ex}{Example}[section]
\theoremstyle{remark}
\newtheorem{remark}{Remark}[section]
\newtheorem*{note}{Note}
\newtheorem*{notation}{Notation}
\theoremstyle{custom}
\newtheorem*{cust}{Definition}
\fancypagestyle{firststyle}
{
   \fancyhead[L]{\textbf{Name:}}
   \fancyhead[R]{\textbf{Worksheet 6: Implicit Differentiation}}
   \fancyfoot[R]{ Thomas Luckner } %{\footnotesize Page \thepage\ of \pageref{LastPage}}
}






\begin{document}
\thispagestyle{firststyle}
\pagestyle{plain}

Thoughts:\\\\
This is one of my favorite sections of Calculus 1. Why? Because it allows me to be lazy! You are not always given a function in terms of just $x$. You may have something like: $xy+y=x$ and you want the derivative with respect to $x$. Now what? Well, what we do is consider $y$ a function of $x$ and approach accordingly! In the example I gave, we have a product rule ($xy$ is a product of 2 functions of $x$) and a single function with respect to $x$. One the right we have a single $x$. Thus, we have the following: $y+xy'+y'=1$. A little confusing, but let me explain. We never do anything we $y$ except make it $y'$ when we are taking its derivative. That is it!\\
Now this should concern you because you learned next to nothing about the derivative! Let's change that by finding $y'$! Let's solve for it. 
\[
xy'+y'=1-y
\]
\[
y'(x+1)=1-y
\]
\[
y'=\dfrac{1-y}{x+1}.
\]
This is useful in answering questions where you need the value of the slope of the tangent line given a point (or an $x$ where you find the $y$-coordinate). That's it! However, this is not its only application.\\
If you can tell me how to find the derivative of this function without using a technique I show you, write it up and present it at math conferences because there is no other way that we know of! Here is the function: $y=x^x$. Weird function, but implicit differentiation helps us here if we use $\ln$ on both sides first!
\[
\ln(y)=x\ln(x).
\]
Now we can use implicit differentiation (chain rule on the left and product rule on the right)!
\[
\dfrac{y'}{y}=\ln(x)+1
\]
\[
\Rightarrow y'=y(\ln(x)+1).
\]
Tada! There is our derivative with respect to $x$. This is the value of implicit differentiation. Another example is for inverse trig functions! Here: $y=\arccos(x)$. Take $\cos$ of both sides. $\cos(y)=x$. Now use implicit! Try it with others!
\newpage
\noindent Problems: Find $y'$ or $\dfrac{dy}{dx}$ for the functions below.
\begin{enumerate}[1.]
\item $x^2+2xy-y^2+x=2$
\begin{solution}
\[
2x+(2y+2xy')-2yy'+1=0 \Rightarrow 2xy'-2yy'=-1-2y-2x \Rightarrow y'=\dfrac{-1-2y-2x}{2x-2y}\]
\end{solution}
\item $x^2+y^2=(2x^2+2y^2-x)^2$
\begin{solution}
\[
2x+2yy'=2(2x^2+2y^2)\cdot (4x+4yy') \Rightarrow 2yy'-8yy'(2x^2+2y^2)=8x(2x^2+2y^2)-2x \Rightarrow y'=\dfrac{8x(2x^2+2y^2)-2x}{2y-8y(2x^2+2y^2)}
\]
\end{solution}
\item $y\sin(2x)=x\cos(2y)$
\begin{solution}
\[
y'\sin(2x)+2y\cos(2x)=\cos(2y)-x\sin(2y)(2y') \Rightarrow y'\sin(2x)+2y'x\sin(2y)=\cos(2y)-2y\cos(2x)
\]
\[
 \Rightarrow y'=\dfrac{\cos(2y)-2y\cos(2x)}{\sin(2x)+2x\sin(2x)}
\]
\end{solution}
\item $x^{2/3}+y^{2/3}=4$
\begin{solution}
\[
(2/3)x^{(-1/3)}+(2/3)y^{(-1/3)}y'=0 \Rightarrow y'=\dfrac{-(2/3)x^{(-1/3)}}{(2/3)y^{(-1/3)}}
\]
\end{solution}
\item $e^{x/y}=x-y$
\begin{solution}
\[
e^{x/y}\cdot \left(\dfrac{y-xy'}{y^2}\right) =1-y' \Rightarrow \dfrac{e^{x/y}}{y}-1 = \dfrac{xy'e^{x/y}}{y^2}-y' \Rightarrow y'=\dfrac{ \dfrac{e^{x/y}}{y}-1}{\dfrac{xe^{x/y}}{y^2}-1}
\]
\end{solution}
\item $\sqrt{x+y}=1+x^2y^2$
\[
(1/2)(x+y)^{(-1/2)}(1+y')=2xy^2+2x^2yy' \Rightarrow (1/2)(x+y)^{(-1/2)}-2xy^2=2x^2yy'-(1/2)y'(x+y)^{(-1/2)}
\]
\[
\Rightarrow y'=\dfrac{ (1/2)(x+y)^{(-1/2)}-2xy^2}{2x^2y-(1/2)(x+y)^{(-1/2)}}
\]
\item $\tan(x-y)=\dfrac{y}{1+x^2}$
\begin{solution}
\[
\sec^2(x-y)\cdot(1-y')=\dfrac{y'(1+x^2)-y(2x)}{(1+x^2)^2} \Rightarrow \sec^2(x-y)+\dfrac{y(2x)}{(1+x^2)^2}=\dfrac{y'}{1+x^2}+y'\sec^2(x-y)
\]
\[
\Rightarrow y'=\dfrac{\sec^2(x-y)+\dfrac{y(2x)}{(1+x^2)^2}}{\dfrac{1}{1+x^2}+\sec^2(x-y)}
\]
\end{solution}
\item $y=\arctan(x)$
\begin{solution}
\[
\Rightarrow \tan(y)=x \Rightarrow \sec^2(y)y'=1 \Rightarrow y'=\dfrac{1}{\sec^2(y)}
\]
\end{solution}
\item $\arctan(x^2y)=x+xy^2$
\begin{solution}
From above,
\[
\dfrac{1}{\sec^2(x^2y)}\cdot (2xy+x^2y')=1+y^2+2xyy' \Rightarrow \dfrac{2xy}{\sec^2(x^2y)}-1-y^2=2xyy'-\dfrac{x^2y'}{\sec^2(x^2y)}
\]
\[
\Rightarrow y'=\dfrac{\dfrac{2xy}{\sec^2(x^2y)}-1-y^2}{2xy-\dfrac{x^2}{\sec^2(x^2y)}}
\]
\end{solution}
\item $y=x^{x^x}$
\begin{solution}
\[
\Rightarrow \ln(y)=x^x\ln(x) \Rightarrow (1/y)y'=(x^x)'\ln(x)+(1/x)(x^x)
\]
Let $y=x^x$ (different $y$ than above).
\[
\ln(y)=x\ln(x) \Rightarrow (1/y)y'=\ln(x)+1 \Rightarrow y'=\dfrac{\ln(x)+1}{(1/x^x)}
\]
Thus, 
\[
(1/y)y'=(x^x)'\ln(x)+(1/x)(x^x) \Rightarrow (1/y)y'=\dfrac{\ln(x)+1}{(1/x^x)}\ln(x)+(1/x)(x^x)
\]
\[
\Rightarrow y'=x^x\left(\dfrac{\ln(x)+1}{(1/x^x)}\ln(x)+(1/x)(x^x)\right)
\]
\end{solution}
\item For 1 - 4 use the points (1,2), $(0,1/2)$, $(\pi/2, \pi/4)$, and $(-3\sqrt{3},1)$ respectively to find tangent lines at the points for the curves.
\begin{solution}
I do not feel the need to find these solutions since the derivatives are the hardest part. What you do with the derivatives is plug in the point and that is your slope. Now use the point and the slope to write an equation.
\end{solution}

\end{enumerate}
\end{document}







