\documentclass[10pt]{article}

\usepackage{enumerate}
\usepackage{amsmath}
\usepackage{amssymb}
\usepackage{amsthm}
\usepackage{array}
\usepackage[all]{xy}
\usepackage{fancyhdr}
\usepackage{euscript}
\usepackage{graphics}
\usepackage{cancel}
\usepackage{fancybox}
\usepackage{tikz}
\usepackage{tikz-3dplot}
\usepackage{pgf}
\usepackage{pgfplots}
\usepackage[all]{xy}
\usepackage{graphicx}
\pgfplotsset{compat=1.14}

\usepackage{pstricks}
\usepackage{pst-plot}

\usepackage{setspace}
\onehalfspacing

\setlength{\oddsidemargin}{.5in}
\setlength{\evensidemargin}{.5in}
\setlength{\textwidth}{6.in}
\setlength{\topmargin}{0in}
\setlength{\headsep}{.20in}
\setlength{\textheight}{8.5in}


\pdfpagewidth 8.5in
 \pdfpageheight 11in


%General
\newcommand{\WW}{\mathbb {W}}
\newcommand{\ZZ}{\mathbb{Z}}
\newcommand{\RR}{\mathbb {R}}
\newcommand{\II}{\mathbb {I}}
\newcommand{\QQ}{\mathbb {Q}}
\newcommand{\CC}{\mathbf C}
\newcommand{\NN}{\mathbb {N}}
\newcommand{\Zn}[1]{\mathbf{Z}/#1\mathbf{Z}}
\newcommand{\Znx}[1]{(\mathbf{Z}/#1\mathbf{Z})^\times}
\newcommand{\X}{\times} 
\newcommand{\set}[2]{\left\{#1 : #2\right\}}          
\newcommand{\sett}[1]{\left\{#1\right\}}                
\newcommand{\nonempty}{\neq\varnothing}
\newcommand{\ds}{\displaystyle}
\newcommand{\abs}[1]{\left| {#1} \right|}
\newcommand{\qedbox}{\rule{2mm}{2mm}}
\renewcommand{\qedsymbol}{\qedbox}											
\newcommand{\aand}{\qquad\hbox{and}\qquad}
\newcommand{\e}{\varepsilon}
\newcommand{\tto}{\rightrightarrows}
\newcommand{\gs}{\geqslant}
\newcommand{\ls}{\leqslant}
\renewcommand{\tilde}{\widetilde}
\renewcommand{\hat}{\widehat}
\newcommand{\norm}[1]{\left\| #1 \right\|}
\newcommand{\md}[3]{#1\equiv#2\;(\mathrm{mod}\;#3)}     
\newcommand{\gen}[1]{\left\langle #1 \right\rangle}
\renewcommand{\Re}{\operatorname{Re}}
\renewcommand{\Im}{\operatorname{Im}}
\newcommand{\zero}{\boldsymbol{0}}

\newcommand{\be}[1]{\textbf{\emph{#1}}}
\newcommand{\hhat}[1]{\hat{\! \hat{#1}}}

\newcommand{\fto}[1]{\xrightarrow{\hspace{4pt} #1 \hspace{4pt}}}
\newcommand{\flto}[1]{\xrightarrow{\quad #1 \quad}}



\newcommand{\dist}{\operatorname{dist}}
\newcommand{\esssup}{\operatorname{ess\:sup}}
\newcommand{\id}{\operatorname{id}}
\newcommand{\card}{\operatorname{card}}

\newcommand{\dmu}{\:\mathrm{d}\mu}
\newcommand{\dm}{\:\mathrm{d}m}
\newcommand{\dx}{\:\mathrm{d}x}
\newcommand{\dt}{\:\mathrm{d}t}
\newcommand{\dz}{\:\mathrm{d}z}
\newcommand{\dtheta}{\:\mathrm{d}\theta}
\newcommand{\dw}{\:\mathrm{d}w}

%Algebra
\newcommand{\Sym}{\operatorname {Sym}}
\newcommand{\Stab}{\operatorname {Stab}}
\newcommand{\M}{\operatorname{M}}
\newcommand{\GL}{\operatorname{GL}}
\newcommand{\PGL}{\operatorname{PGL}}
\newcommand{\SL}{\operatorname{SL}}
\newcommand{\PSL}{\operatorname{PSL}}
\newcommand{\Heis}{\operatorname{Heis}}
\newcommand{\Aff}{\operatorname{Aff}}
\newcommand{\Aut}{\operatorname{Aut}}
\newcommand{\image}{\operatorname{im}}
\newcommand{\Syl}[2]{\operatorname{\emph{Syl}}_{#1}\left(#2\right)}
\newcommand{\Hom}{\operatorname{Hom}}
\newcommand{\Tor}{\operatorname{Tor}}
\newcommand{\Gal}{\operatorname{Gal}}
\newcommand{\ch}{\operatorname{ch}}
\newcommand{\rad}{\operatorname{rad}}
\newcommand{\iso}{\cong}
\newcommand{\normal}{\unlhd}
\newcommand{\semi}{\rtimes}
\newcommand{\Nm}{\operatorname {N}}
\newcommand{\Tr}{\operatorname {Tr}}
\newcommand{\disc}{\operatorname {disc}}








%Euler Script Characters
\newcommand{\esa}{\EuScript{A}}
\newcommand{\esb}{\EuScript{B}}
\newcommand{\esc}{\EuScript{C}}
\newcommand{\esd}{\EuScript{D}}
\newcommand{\ese}{\EuScript{E}}
\newcommand{\esf}{\EuScript{F}}
\newcommand{\esg}{\EuScript{G}}
\newcommand{\esh}{\EuScript{H}}
\newcommand{\esi}{\EuScript{I}}
\newcommand{\esj}{\EuScript{J}}
\newcommand{\esk}{\EuScript{K}}
\newcommand{\esl}{\EuScript{L}}
\newcommand{\esm}{\EuScript{M}}
\newcommand{\esn}{\EuScript{N}}
\newcommand{\eso}{\EuScript{O}}
\newcommand{\esp}{\EuScript{P}}
\newcommand{\esq}{\EuScript{Q}}
\newcommand{\esr}{\EuScript{R}}
\newcommand{\ess}{\EuScript{S}}
\newcommand{\est}{\EuScript{T}}
\newcommand{\esu}{\EuScript{U}}
\newcommand{\esv}{\EuScript{V}}
\newcommand{\esw}{\EuScript{W}}
\newcommand{\esx}{\EuScript{X}}
\newcommand{\esy}{\EuScript{Y}}
\newcommand{\esz}{\EuScript{Z}}

%Calligraphic Characters
\newcommand{\cala}{\mathcal{A}}
\newcommand{\calb}{\mathcal{B}}
\newcommand{\calc}{\mathcal{C}}
\newcommand{\cald}{\mathcal{D}}
\newcommand{\cale}{\mathcal{E}}
\newcommand{\calf}{\mathcal{F}}
\newcommand{\calg}{\mathcal{G}}
\newcommand{\calh}{\mathcal{H}}
\newcommand{\cali}{\mathcal{I}}
\newcommand{\calj}{\mathcal{J}}
\newcommand{\calk}{\mathcal{K}}
\newcommand{\call}{\mathcal{L}}
\newcommand{\calm}{\mathcal{M}}
\newcommand{\caln}{\mathcal{N}}
\newcommand{\calo}{\mathcal{O}}
\newcommand{\calp}{\mathcal{P}}
\newcommand{\calq}{\mathcal{Q}}
\newcommand{\calr}{\mathcal{R}}
\newcommand{\cals}{\mathcal{S}}
\newcommand{\calt}{\mathcal{T}}
\newcommand{\calu}{\mathcal{U}}
\newcommand{\calv}{\mathcal{V}}
\newcommand{\calw}{\mathcal{W}}
\newcommand{\calx}{\mathcal{X}}
\newcommand{\caly}{\mathcal{Y}}
\newcommand{\calz}{\mathcal{Z}}

%Gothic Characters
\newcommand{\fraka}{\mathfrak{a}}
\newcommand{\frakb}{\mathfrak{b}}
\newcommand{\frakc}{\mathfrak{c}}
\newcommand{\frakd}{\mathfrak{d}}
\newcommand{\frake}{\mathfrak{e}}
\newcommand{\frakf}{\mathfrak{f}}
\newcommand{\frakg}{\mathfrak{g}}
\newcommand{\frakh}{\mathfrak{h}}
\newcommand{\fraki}{\mathfrak{i}}
\newcommand{\frakj}{\mathfrak{j}}
\newcommand{\frakk}{\mathfrak{k}}
\newcommand{\frakl}{\mathfrak{l}}
\newcommand{\frakm}{\mathfrak{m}}
\newcommand{\frakn}{\mathfrak{n}}
\newcommand{\frako}{\mathfrak{o}}
\newcommand{\frakp}{\mathfrak{p}}
\newcommand{\frakq}{\mathfrak{q}}
\newcommand{\frakr}{\mathfrak{r}}
\newcommand{\fraks}{\mathfrak{s}}
\newcommand{\frakt}{\mathfrak{t}}
\newcommand{\fraku}{\mathfrak{u}}
\newcommand{\frakv}{\mathfrak{v}}
\newcommand{\frakw}{\mathfrak{w}}
\newcommand{\frakx}{\mathfrak{x}}
\newcommand{\fraky}{\mathfrak{y}}
\newcommand{\frakz}{\mathfrak{z}}

\newcommand{\frakA}{\mathfrak{A}}
\newcommand{\frakB}{\mathfrak{B}}
\newcommand{\frakC}{\mathfrak{C}}
\newcommand{\frakD}{\mathfrak{D}}
\newcommand{\frakE}{\mathfrak{E}}
\newcommand{\frakF}{\mathfrak{F}}
\newcommand{\frakG}{\mathfrak{G}}
\newcommand{\frakH}{\mathfrak{H}}
\newcommand{\frakI}{\mathfrak{I}}
\newcommand{\frakJ}{\mathfrak{J}}
\newcommand{\frakK}{\mathfrak{K}}
\newcommand{\frakL}{\mathfrak{L}}
\newcommand{\frakM}{\mathfrak{M}}
\newcommand{\frakN}{\mathfrak{N}}
\newcommand{\frakO}{\mathfrak{O}}
\newcommand{\frakP}{\mathfrak{P}}
\newcommand{\frakQ}{\mathfrak{Q}}
\newcommand{\frakR}{\mathfrak{R}}
\newcommand{\frakS}{\mathfrak{S}}
\newcommand{\frakT}{\mathfrak{T}}
\newcommand{\frakU}{\mathfrak{U}}
\newcommand{\frakV}{\mathfrak{V}}
\newcommand{\frakW}{\mathfrak{W}}
\newcommand{\frakX}{\mathfrak{X}}
\newcommand{\frakY}{\mathfrak{Y}}
\newcommand{\frakZ}{\mathfrak{Z}}

%Lowercase Bold Letters
\newcommand{\bfa}{\mathbf{a}}
\newcommand{\bfb}{\mathbf{b}}
\newcommand{\bfc}{\mathbf{c}}
\newcommand{\bfd}{\mathbf{d}}
\newcommand{\bfe}{\mathbf{e}}
\newcommand{\bff}{\mathbf{f}}
\newcommand{\bfg}{\mathbf{g}}
\newcommand{\bfh}{\mathbf{h}}
\newcommand{\bfi}{\mathbf{i}}
\newcommand{\bfj}{\mathbf{j}}
\newcommand{\bfk}{\mathbf{k}}
\newcommand{\bfl}{\mathbf{l}}
\newcommand{\bfm}{\mathbf{m}}
\newcommand{\bfn}{\mathbf{n}}
\newcommand{\bfo}{\mathbf{o}}
\newcommand{\bfp}{\mathbf{p}}
\newcommand{\bfq}{\mathbf{q}}
\newcommand{\bfr}{\mathbf{r}}
\newcommand{\bfs}{\mathbf{s}}
\newcommand{\bft}{\mathbf{t}}
\newcommand{\bfu}{\mathbf{u}}
\newcommand{\bfv}{\mathbf{v}}
\newcommand{\bfw}{\mathbf{w}}
\newcommand{\bfx}{\mathbf{x}}
\newcommand{\bfy}{\mathbf{y}}
\newcommand{\bfz}{\mathbf{z}}




%Customized Theorem Environments
\newtheoremstyle%
{custom}%
{}%                         Space above
{}%													Space below
{}%													Body font
{}%                         Indent amount
{}%                         Theorem head font
{.}%                        Punctuation after heading
{ }%                        Space after heading
{\thmname{}%                Additional specifications for theorem head
\thmnumber{}%
\thmnote{\bfseries #3}}%

\newtheoremstyle%
{Theorem}%
{}%
{}%
{\itshape}%
{}%
{}%
{.}%
{ }%
{\thmname{\bfseries #1}%
\thmnumber{\;\bfseries #2}%
\thmnote{\;(\bfseries #3)}}%

%Theorem Environments
\theoremstyle{Theorem}
\newtheorem{theorem}{Theorem}[section]
\newtheorem{cor}{Corollary}[section]
\newtheorem{lemma}{Lemma}[section]
\newtheorem{prop}{Proposition}[section]
\newtheorem*{nonumthm}{Theorem}
\newtheorem*{nonumprop}{Proposition}
\theoremstyle{definition}
\newtheorem{definition}{Definition}[section]
\newtheorem*{answer}{Answer}
\newtheorem*{solution}{Solution}
\newtheorem*{nonumdfn}{Definition}
\newtheorem*{nonumex}{Example}
\newtheorem{ex}{Example}[section]
\theoremstyle{remark}
\newtheorem{remark}{Remark}[section]
\newtheorem*{note}{Note}
\newtheorem*{notation}{Notation}
\theoremstyle{custom}
\newtheorem*{cust}{Definition}
\fancypagestyle{firststyle}
{
   \fancyhead[L]{\textbf{Name:}}
   \fancyhead[R]{\textbf{Worksheet 9: L'Hospital's Rule}}
   \fancyfoot[R]{ Thomas Luckner } %{\footnotesize Page \thepage\ of \pageref{LastPage}}
}






\begin{document}
\thispagestyle{firststyle}
\pagestyle{plain}

Thoughts:\\\\
This section is short and sweet. You are hopefully familiar with what a limit can tell us aboout a function. Thus, finding limits can be very important. Now some limits are not very obvious. For example, consider this function: 
\[
f(x)=\dfrac{e^x}{x+1}.
\]
Most of the function is well defined and not hard to find limits of. However, going to infinity might pose a problem! Let's see this mathematically.
\[
\lim_{x\rightarrow \infty}\dfrac{e^x}{x+1}=\dfrac{\infty}{\infty}.
\]
Well we have the same thing on top and bottom, so this is 1 right? WRONG! Remember infinity is not an actual number so we have an issue here! Now let's reason through this problem a little better. The big question here is what infinity is bigger? In other words, which grows faster; $e^x$ or $x+1$? Since $e^x$ is exponential, it is certainly faster than $x+1$. Thus, this limit is in fact infinity! Now that we can reason this out without the mathematics, let's consider what we just said. The words "grow faster" are very important here. When we say grow faster, we are actually talking about the rate of the function. This word rate should ring a bell. Rate is commonly used to refer to derivative! That is exactly what we can do here! Since we care about which grows faster, we can take the derivative of each of them independently and divide them, and the limit will remain unchanged! Let's try that!
\[
\lim_{x\rightarrow \infty}\dfrac{e^x}{x+1}=\dfrac{\infty}{\infty}=\lim_{x\rightarrow \infty}\dfrac{e^x}{1}=\infty.
\]
Exactly as we expected! This is not the only case where we can consider this train of thought. Let's consider another function.
\[
f(x)=\dfrac{\ln(x)}{x-1}
\]
Now this function is undefined at $x=1$, but there is something a little more interesting going on there. Clearly, both top and bottom are tending toward 0 when $x$ approaches 1. The question is, once again, which function does it faster! We can use the same concept as before with using the derivative.
\[
\lim_{x\rightarrow 1}\dfrac{\ln(x)}{x-1}=\lim_{x\rightarrow 1}\dfrac{(1/x)}{1}=1.
\]
Now does this match our logic is the question. If you consider the slope of both functions at the point $x=1$, you can see they are growing/decaying at the same exact speed! So this logic tells us we have what we want.\\
\newpage This concept is what is known as L'Hospital's Rule. Here we formalize it with the definition of these forms and then the rule.
\begin{definition}[Indeterminant Forms]
A limit is in indeterminant form if the limit of $x$ approaching some $a$ is either $\dfrac{0}{0}$ or $\dfrac{\pm\infty}{\pm\infty}$.\\
NOTE: $\pm\infty \cdot 0=\dfrac{\pm\infty}{\pm\infty}=\dfrac{0}{0}$.
\end{definition}
The note here is due to keep, change, flip! Here is an example: $e^x\cdot (1/x)$ when $x$ is going to infinity is $\infty \cdot 0$. If you rewrite this as $\dfrac{e^x}{(1/x)^{-1}}=\dfrac{e^x}{x}$ you get an indeterminant form.
\begin{theorem}[L'Hospital's Rule]
If a function, $h(x)=\dfrac{f(x)}{g(x)}$ is indeterminant form when $x$ approaches $a$ then
\[ 
\lim_{x\rightarrow a}\dfrac{f(x)}{g(x)}=\lim_{x\rightarrow a}\dfrac{f'(x)}{g'(x)}.
\]
\end{theorem}
The big thing here is that you can use this rule on the same limit over and over and over again until you get what you want! For example, $\dfrac{e^x}{x^2}$ you can use it twice to get what you want when going off to infinity!\\
There are a few more indeterminant forms which are much less common, but are worth mentioning and one is $\infty - \infty$. You can see the question is which is faster still, but we need it in fraction form to apply L'Hospital's Rule. This is how we do it. The function will pose itself so that this is possible, but it is not always possible! A few other forms are $0^0$, $\infty^0$, and $1^\infty$. You can use the natural logarithm to figure these out and up with
\[
y=f(x)^{g(x)} \Rightarrow \ln(y)=g(x)\ln(f(x))
\]
which is now in a product indeterminant form given one of the forms listed is true.

\newpage
\noindent Problems:\\
\begin{enumerate}[1.]
\item $\ds \lim_{x\rightarrow 1}\dfrac{x^3-2x^2+1}{x^3-1}$
\begin{solution}
Indeterminant form of 0's. 
\[
\lim_{x\rightarrow 1}\dfrac{x^3-2x^2+1}{x^3-1}=\lim_{x\rightarrow 1}\dfrac{3x^2-4x}{3x^2}=\dfrac{-1}{3}
\]
\end{solution}
\item $\ds \lim_{x\rightarrow 0}\dfrac{x^2}{1-\cos(x)}$
\begin{solution}
Indeterminant form of 0's. 
\[
\lim_{x\rightarrow 0}\dfrac{x^2}{1-\cos(x)}=\lim_{x\rightarrow 0} \dfrac{2x}{\sin(x)}
\]
Once again 0's. 
\[
\lim_{x\rightarrow 0} \dfrac{2x}{\sin(x)}=\lim_{x\rightarrow 0}\dfrac{2}{\cos(x)}=2
\]
\end{solution}
\item $\ds \lim_{x\rightarrow \infty}\dfrac{(\ln(x))^2}{x}$
\begin{solution}
Indeterminant form of $\infty$'s.
\[
\lim_{x\rightarrow \infty}\dfrac{(\ln(x))^2}{x}=\lim_{x\rightarrow \infty} \dfrac{2\ln(x)(1/x)}{1}=\lim_{x\rightarrow \infty} \dfrac{2\ln(x)}{x}=\lim_{x\rightarrow \infty} \dfrac{2}{x}=0
\]
\end{solution}
\item $\ds \lim_{x\rightarrow \infty} x^3e^{-x^2}$
\begin{solution}
Not an obvious indeterminant form. 
\[
 \lim_{x\rightarrow \infty} x^3e^{-x^2}=\lim_{x\rightarrow \infty}\dfrac{x^3}{e^{x^2}}
\]
Now we see indeterminant $\infty$'s. You may notice $e$ grows faster than any linear function. Thus, this is 0. If not, you'll take derivatives until the numerator is constant and an $e$ still exists on the bottom. Thus 0.
\end{solution}
\item $\ds \lim_{x\rightarrow 0}\csc(x)-\cot(x)$
\begin{solution}
Not obvious at all! 
\[
\lim_{x\rightarrow 0}\csc(x)-\cot(x)=\lim_{x\rightarrow 0} \dfrac{1}{\sin(x)}-\dfrac{\cos(x)}{\sin(x)}=\lim_{x\rightarrow 0} \dfrac{1-\cos(x)}{\sin(x)}
\]
Now you see the indeterminant form of 0's.
\[
\lim_{x\rightarrow 0} \dfrac{1-\cos(x)}{\sin(x)}=\lim_{x\rightarrow 0}\dfrac{\sin(x)}{\cos(x)}=1
\]
\end{solution}
\item $\ds \lim_{x\rightarrow \infty} x^{\ln(2)/(1+\ln(x))}$
\begin{solution}
This is a tricky one. There is an issue since you get $\infty^0$. Is it 1 or $\infty$? That is the problem! Thus, we need to get indeterminant form. Let's try the following:
\[
 \lim_{x\rightarrow \infty} x^{\ln(2)/(1+\ln(x))}=y \Rightarrow \ln(y)=\lim_{x\rightarrow \infty}\dfrac{\ln(2)}{1+\ln(x)}\ln(x)
 \]
 Here you see the form of $\infty$'s.
 \[
\ln(y)= \lim_{x\rightarrow \infty}\dfrac{\ln(2)}{1+\ln(x)}\ln(x)= \lim_{x\rightarrow \infty}\ln(2)=\ln(2)
 \]
 Now we go back to find $y$, our original limit. 
 \[
 \ln(y)=\ln(2) \Rightarrow y=2
 \]
\end{solution}
\end{enumerate}
\end{document}







