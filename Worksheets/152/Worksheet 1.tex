\documentclass[10pt]{article}

\usepackage{enumerate}
\usepackage{amsmath}
\usepackage{amssymb}
\usepackage{amsthm}
\usepackage{array}
\usepackage[all]{xy}
\usepackage{fancyhdr}
\usepackage{euscript}
\usepackage{graphics}
\usepackage{cancel}
\usepackage{fancybox}
\usepackage{tikz}
\usepackage{tikz-3dplot}
\usepackage{pgf}
\usepackage{pgfplots}
\usepackage[all]{xy}
\usepackage{graphicx}
\pgfplotsset{compat=1.14}

\usepackage{pstricks}
\usepackage{pst-plot}

\usepackage{setspace}
\onehalfspacing

\setlength{\oddsidemargin}{.5in}
\setlength{\evensidemargin}{.5in}
\setlength{\textwidth}{6.in}
\setlength{\topmargin}{0in}
\setlength{\headsep}{.20in}
\setlength{\textheight}{8.5in}


\pdfpagewidth 8.5in
 \pdfpageheight 11in


%General
\newcommand{\WW}{\mathbb {W}}
\newcommand{\ZZ}{\mathbb{Z}}
\newcommand{\RR}{\mathbb {R}}
\newcommand{\II}{\mathbb {I}}
\newcommand{\QQ}{\mathbb {Q}}
\newcommand{\CC}{\mathbf C}
\newcommand{\NN}{\mathbb {N}}
\newcommand{\Zn}[1]{\mathbf{Z}/#1\mathbf{Z}}
\newcommand{\Znx}[1]{(\mathbf{Z}/#1\mathbf{Z})^\times}
\newcommand{\X}{\times} 
\newcommand{\set}[2]{\left\{#1 : #2\right\}}          
\newcommand{\sett}[1]{\left\{#1\right\}}                
\newcommand{\nonempty}{\neq\varnothing}
\newcommand{\ds}{\displaystyle}
\newcommand{\abs}[1]{\left| {#1} \right|}
\newcommand{\qedbox}{\rule{2mm}{2mm}}
\renewcommand{\qedsymbol}{\qedbox}											
\newcommand{\aand}{\qquad\hbox{and}\qquad}
\newcommand{\e}{\varepsilon}
\newcommand{\tto}{\rightrightarrows}
\newcommand{\gs}{\geqslant}
\newcommand{\ls}{\leqslant}
\renewcommand{\tilde}{\widetilde}
\renewcommand{\hat}{\widehat}
\newcommand{\norm}[1]{\left\| #1 \right\|}
\newcommand{\md}[3]{#1\equiv#2\;(\mathrm{mod}\;#3)}     
\newcommand{\gen}[1]{\left\langle #1 \right\rangle}
\renewcommand{\Re}{\operatorname{Re}}
\renewcommand{\Im}{\operatorname{Im}}
\newcommand{\zero}{\boldsymbol{0}}

\newcommand{\be}[1]{\textbf{\emph{#1}}}
\newcommand{\hhat}[1]{\hat{\! \hat{#1}}}

\newcommand{\fto}[1]{\xrightarrow{\hspace{4pt} #1 \hspace{4pt}}}
\newcommand{\flto}[1]{\xrightarrow{\quad #1 \quad}}



\newcommand{\dist}{\operatorname{dist}}
\newcommand{\esssup}{\operatorname{ess\:sup}}
\newcommand{\id}{\operatorname{id}}
\newcommand{\card}{\operatorname{card}}

\newcommand{\dmu}{\:\mathrm{d}\mu}
\newcommand{\dm}{\:\mathrm{d}m}
\newcommand{\dx}{\:\mathrm{d}x}
\newcommand{\dt}{\:\mathrm{d}t}
\newcommand{\dz}{\:\mathrm{d}z}
\newcommand{\dtheta}{\:\mathrm{d}\theta}
\newcommand{\dw}{\:\mathrm{d}w}

%Algebra
\newcommand{\Sym}{\operatorname {Sym}}
\newcommand{\Stab}{\operatorname {Stab}}
\newcommand{\M}{\operatorname{M}}
\newcommand{\GL}{\operatorname{GL}}
\newcommand{\PGL}{\operatorname{PGL}}
\newcommand{\SL}{\operatorname{SL}}
\newcommand{\PSL}{\operatorname{PSL}}
\newcommand{\Heis}{\operatorname{Heis}}
\newcommand{\Aff}{\operatorname{Aff}}
\newcommand{\Aut}{\operatorname{Aut}}
\newcommand{\image}{\operatorname{im}}
\newcommand{\Syl}[2]{\operatorname{\emph{Syl}}_{#1}\left(#2\right)}
\newcommand{\Hom}{\operatorname{Hom}}
\newcommand{\Tor}{\operatorname{Tor}}
\newcommand{\Gal}{\operatorname{Gal}}
\newcommand{\ch}{\operatorname{ch}}
\newcommand{\rad}{\operatorname{rad}}
\newcommand{\iso}{\cong}
\newcommand{\normal}{\unlhd}
\newcommand{\semi}{\rtimes}
\newcommand{\Nm}{\operatorname {N}}
\newcommand{\Tr}{\operatorname {Tr}}
\newcommand{\disc}{\operatorname {disc}}








%Euler Script Characters
\newcommand{\esa}{\EuScript{A}}
\newcommand{\esb}{\EuScript{B}}
\newcommand{\esc}{\EuScript{C}}
\newcommand{\esd}{\EuScript{D}}
\newcommand{\ese}{\EuScript{E}}
\newcommand{\esf}{\EuScript{F}}
\newcommand{\esg}{\EuScript{G}}
\newcommand{\esh}{\EuScript{H}}
\newcommand{\esi}{\EuScript{I}}
\newcommand{\esj}{\EuScript{J}}
\newcommand{\esk}{\EuScript{K}}
\newcommand{\esl}{\EuScript{L}}
\newcommand{\esm}{\EuScript{M}}
\newcommand{\esn}{\EuScript{N}}
\newcommand{\eso}{\EuScript{O}}
\newcommand{\esp}{\EuScript{P}}
\newcommand{\esq}{\EuScript{Q}}
\newcommand{\esr}{\EuScript{R}}
\newcommand{\ess}{\EuScript{S}}
\newcommand{\est}{\EuScript{T}}
\newcommand{\esu}{\EuScript{U}}
\newcommand{\esv}{\EuScript{V}}
\newcommand{\esw}{\EuScript{W}}
\newcommand{\esx}{\EuScript{X}}
\newcommand{\esy}{\EuScript{Y}}
\newcommand{\esz}{\EuScript{Z}}

%Calligraphic Characters
\newcommand{\cala}{\mathcal{A}}
\newcommand{\calb}{\mathcal{B}}
\newcommand{\calc}{\mathcal{C}}
\newcommand{\cald}{\mathcal{D}}
\newcommand{\cale}{\mathcal{E}}
\newcommand{\calf}{\mathcal{F}}
\newcommand{\calg}{\mathcal{G}}
\newcommand{\calh}{\mathcal{H}}
\newcommand{\cali}{\mathcal{I}}
\newcommand{\calj}{\mathcal{J}}
\newcommand{\calk}{\mathcal{K}}
\newcommand{\call}{\mathcal{L}}
\newcommand{\calm}{\mathcal{M}}
\newcommand{\caln}{\mathcal{N}}
\newcommand{\calo}{\mathcal{O}}
\newcommand{\calp}{\mathcal{P}}
\newcommand{\calq}{\mathcal{Q}}
\newcommand{\calr}{\mathcal{R}}
\newcommand{\cals}{\mathcal{S}}
\newcommand{\calt}{\mathcal{T}}
\newcommand{\calu}{\mathcal{U}}
\newcommand{\calv}{\mathcal{V}}
\newcommand{\calw}{\mathcal{W}}
\newcommand{\calx}{\mathcal{X}}
\newcommand{\caly}{\mathcal{Y}}
\newcommand{\calz}{\mathcal{Z}}

%Gothic Characters
\newcommand{\fraka}{\mathfrak{a}}
\newcommand{\frakb}{\mathfrak{b}}
\newcommand{\frakc}{\mathfrak{c}}
\newcommand{\frakd}{\mathfrak{d}}
\newcommand{\frake}{\mathfrak{e}}
\newcommand{\frakf}{\mathfrak{f}}
\newcommand{\frakg}{\mathfrak{g}}
\newcommand{\frakh}{\mathfrak{h}}
\newcommand{\fraki}{\mathfrak{i}}
\newcommand{\frakj}{\mathfrak{j}}
\newcommand{\frakk}{\mathfrak{k}}
\newcommand{\frakl}{\mathfrak{l}}
\newcommand{\frakm}{\mathfrak{m}}
\newcommand{\frakn}{\mathfrak{n}}
\newcommand{\frako}{\mathfrak{o}}
\newcommand{\frakp}{\mathfrak{p}}
\newcommand{\frakq}{\mathfrak{q}}
\newcommand{\frakr}{\mathfrak{r}}
\newcommand{\fraks}{\mathfrak{s}}
\newcommand{\frakt}{\mathfrak{t}}
\newcommand{\fraku}{\mathfrak{u}}
\newcommand{\frakv}{\mathfrak{v}}
\newcommand{\frakw}{\mathfrak{w}}
\newcommand{\frakx}{\mathfrak{x}}
\newcommand{\fraky}{\mathfrak{y}}
\newcommand{\frakz}{\mathfrak{z}}

\newcommand{\frakA}{\mathfrak{A}}
\newcommand{\frakB}{\mathfrak{B}}
\newcommand{\frakC}{\mathfrak{C}}
\newcommand{\frakD}{\mathfrak{D}}
\newcommand{\frakE}{\mathfrak{E}}
\newcommand{\frakF}{\mathfrak{F}}
\newcommand{\frakG}{\mathfrak{G}}
\newcommand{\frakH}{\mathfrak{H}}
\newcommand{\frakI}{\mathfrak{I}}
\newcommand{\frakJ}{\mathfrak{J}}
\newcommand{\frakK}{\mathfrak{K}}
\newcommand{\frakL}{\mathfrak{L}}
\newcommand{\frakM}{\mathfrak{M}}
\newcommand{\frakN}{\mathfrak{N}}
\newcommand{\frakO}{\mathfrak{O}}
\newcommand{\frakP}{\mathfrak{P}}
\newcommand{\frakQ}{\mathfrak{Q}}
\newcommand{\frakR}{\mathfrak{R}}
\newcommand{\frakS}{\mathfrak{S}}
\newcommand{\frakT}{\mathfrak{T}}
\newcommand{\frakU}{\mathfrak{U}}
\newcommand{\frakV}{\mathfrak{V}}
\newcommand{\frakW}{\mathfrak{W}}
\newcommand{\frakX}{\mathfrak{X}}
\newcommand{\frakY}{\mathfrak{Y}}
\newcommand{\frakZ}{\mathfrak{Z}}

%Lowercase Bold Letters
\newcommand{\bfa}{\mathbf{a}}
\newcommand{\bfb}{\mathbf{b}}
\newcommand{\bfc}{\mathbf{c}}
\newcommand{\bfd}{\mathbf{d}}
\newcommand{\bfe}{\mathbf{e}}
\newcommand{\bff}{\mathbf{f}}
\newcommand{\bfg}{\mathbf{g}}
\newcommand{\bfh}{\mathbf{h}}
\newcommand{\bfi}{\mathbf{i}}
\newcommand{\bfj}{\mathbf{j}}
\newcommand{\bfk}{\mathbf{k}}
\newcommand{\bfl}{\mathbf{l}}
\newcommand{\bfm}{\mathbf{m}}
\newcommand{\bfn}{\mathbf{n}}
\newcommand{\bfo}{\mathbf{o}}
\newcommand{\bfp}{\mathbf{p}}
\newcommand{\bfq}{\mathbf{q}}
\newcommand{\bfr}{\mathbf{r}}
\newcommand{\bfs}{\mathbf{s}}
\newcommand{\bft}{\mathbf{t}}
\newcommand{\bfu}{\mathbf{u}}
\newcommand{\bfv}{\mathbf{v}}
\newcommand{\bfw}{\mathbf{w}}
\newcommand{\bfx}{\mathbf{x}}
\newcommand{\bfy}{\mathbf{y}}
\newcommand{\bfz}{\mathbf{z}}




%Customized Theorem Environments
\newtheoremstyle%
{custom}%
{}%                         Space above
{}%													Space below
{}%													Body font
{}%                         Indent amount
{}%                         Theorem head font
{.}%                        Punctuation after heading
{ }%                        Space after heading
{\thmname{}%                Additional specifications for theorem head
\thmnumber{}%
\thmnote{\bfseries #3}}%

\newtheoremstyle%
{Theorem}%
{}%
{}%
{\itshape}%
{}%
{}%
{.}%
{ }%
{\thmname{\bfseries #1}%
\thmnumber{\;\bfseries #2}%
\thmnote{\;(\bfseries #3)}}%

%Theorem Environments
\theoremstyle{Theorem}
\newtheorem{theorem}{Theorem}[section]
\newtheorem{cor}{Corollary}[section]
\newtheorem{lemma}{Lemma}[section]
\newtheorem{prop}{Proposition}[section]
\newtheorem*{nonumthm}{Theorem}
\newtheorem*{nonumprop}{Proposition}
\theoremstyle{definition}
\newtheorem{definition}{Definition}[section]
\newtheorem*{answer}{Answer}
\newtheorem*{nonumdfn}{Definition}
\newtheorem*{nonumex}{Example}
\newtheorem{ex}{Example}[section]
\theoremstyle{remark}
\newtheorem{remark}{Remark}[section]
\newtheorem*{note}{Note}
\newtheorem*{notation}{Notation}
\theoremstyle{custom}
\newtheorem*{cust}{Definition}
\fancypagestyle{firststyle}
{
   \fancyhead[L]{\textbf{Name:}}
   \fancyhead[R]{\textbf{Worksheet 1: Integration by parts, trig identities, and trig substitution}}
   \fancyfoot[R]{ Thomas Luckner } %{\footnotesize Page \thepage\ of \pageref{LastPage}}
}






\begin{document}
\thispagestyle{firststyle}
\pagestyle{plain}


Thoughts:\\
Integration by parts: Always for a product of 2 functions! There really two cases where this is needed.\\
One function when taking the derivative \textbf{EVENTUALLY} either becomes constant or gets "absorbed" by the other function.\\
ex: $\int x^3e^x \dx$. The $x^3$ will become a constant when by parts is applied 3 times while $e^x$ remains unchanged.\\
ex: $\int x\ln(x) \dx$. In this case, the derivative of $\ln(x)$ is $1/x$. Since the antiderivative of $x$ is $x^2/2$, the integral will be of $x/2$. This is an example of what I mean with "absorbing".\\\\
Trig Identities: It is best to start this with an example. Consider the following integral:
$$\int \sin^5(x) \dx.$$
Your first instinct at this point should be $u$-sub.  You'll quickly find that $\cos(x)$ is not in this problem so $u$-sub is a dead end. Now what? Well this leaves us no choice but to use a trig identity! The question now becomes which one? The power here should be the hint.  We know things about $\sin^2(x)$! This is seen in the identity $\sin^2(x)+\cos^2(x)=1$. Thus, $\sin^2(x)=1-\cos^2(x)$.  Before we make use of this, we need to remember that a trig identity is used to help induce a $u$-substitution! Now the question is how do I apply this? Let's see where $\sin^2(x)$ makes its appearance in the problem.
$$\int \sin^5(x) \dx=\int (\sin^2(x))(\sin^2(x))\sin(x) \dx.$$
Now what? Well we see $\sin^2(x)$ twice, so let's apply our identity twice!
$$\int (1-\cos^2(x))^2\sin(x) \dx.$$
Now is where we stop and think about the $u$-substitution we are trying to induce. Notice we have a product of 2 functions. Thus, the $u$ will live inside one of these functions. $\sin(x)$ is too simple to be a helpful $u$. Thus, it must be in our $du$. That tells us we need something that gives us a constant times $\sin(x)$ as our $du$! That leaves us no choice, but $u=\cos(x)$. So we are left with 
$$\int (1-\cos^2(x))^2\sin(x) \dx=-\int (1-u^2)^2 \text{ d}u .$$
This problem is now an example of a basic algebra manipulation to use power rule of antiderivatives!\\ This gives us a general idea for how to approach trig identity problems! First see if $u$-sub works right away. If not, we make it work with trig identities!
\newpage
Trig substitution: This is a difficult topic. Many approach this topic in a memorization way which I find unvaluable.  Let's make sense of what is going on with an example.  
$$\int \dfrac{\sqrt{25x^2-4}}{x} \dx.$$
Very intimidating problem at first glance.  $u$-sub does not work. Nothing straight and simple about this right away. Thus, we are left to trig substitution! The idea for trig sub stems from the right triangle trig from precalculus.  Notice our difference is of two squares. This should be a \textbf{HUGE} hint to use trig substitution! Here is why.
We have a lot of identities which are useful to us with squares like this problem has with the difference of 2 squares, but they always include a 1. Let's derive that!\\
 So, we have $25x^2-4$. Our goal here is to achieve a constant 1. We can do that by dividing by 4! Thus, we have $1/4\left(\dfrac{25x^2}{4}-1\right)$. Now we remember a substitution that is a trig function squared minus 1. This is $\tan^2(\theta)+1=\sec^2(\theta)$ since the manipulation gives us $\tan^2(\theta)=\sec^2(\theta)-1$. We have identified our identity and we notice $\sec^2(\theta)=\dfrac{25x^2}{4}$ in our instance.  Thus, $\sec(\theta)=\dfrac{5x}{2}$ and $x=\dfrac{2\sec(\theta)}{5}$. Let's ignore the $dx$ for now and see what we have with the root.
 $$\sqrt{25x^2-4}=\sqrt{(4)\left(\dfrac{25x^2}{4}-1\right)}=\sqrt{\sec^2(\theta)-1}=|\tan(\theta)|.$$ 
 Now what? Well we left out $dx$! $dx=\dfrac{2}{5}\sec(x)\tan(x) \text{ d} \theta$.
 That gives us
 $$\int \dfrac{\sqrt{25x^2-4}}{x} \dx=\int \dfrac{|\tan(\theta)|}{\dfrac{2}{5}\sec(\theta)}\left(\dfrac{2}{5}\sec(\theta)\tan(\theta)\right) \text{ d}\theta.$$
 Now we can apply the before section on trig identities to finish this problem. Now at the end we need to translate things back into x's! That is where right triangle trig comes in!\\
We can use our $x$ substitution to make the triangle! $\sec(\theta)$ tells us that  $a=2$, $b=\sqrt{25x^2-4}$ and $c=5x$ in the below triangle. Our goal is to use this info to put a trig function in! Let's make it happen using triangles!  
\begin{center}
\begin{tikzpicture}[scale=.9]
	\draw[-] (0,0) -- (4,0) node[pos=0.5, below]{$2$};
	\draw[-] (0,0) -- (0,4) node[pos=0.5, left]{$\sqrt{25x^2-4}$};
	\draw[-] (4,0) -- (0,4) node[pos=0.5, right]{$5x$};
	
\end{tikzpicture}
\end{center}
 Now we have our triangle which gives us the value in terms of x for all trig functions! This is a good basis for trig sustitution.\\\\
Now some problems.
\begin{enumerate}[1.]
\item $\ds \int_0^{2\pi}t^2\sin(2t) \text{ d}t$
\item $\ds \int_1^2\dfrac{(\ln(x))^2}{x^3} \dx$
\item $\ds \int x\tan^2(x) \dx$
\item $\ds \int xe^{-3x}\dx$
\item $\ds \int \sin^2(x) \cos^2(x) \dx$ (Hint: half-angle identity)
\item $\ds \int \cos(\theta)\cos^5(\sin(\theta)) \text{ d}\theta$
\item $\ds \int \tan^3(x) \sec(x) \dx$
\item $\ds \int \dfrac{x}{\sqrt{36-x^2}}\dx$
\item $\ds \int \dfrac{\dx}{(x^2+1)^2}$
\item $\ds \int \dfrac{x^2}{(3+4x-4x^2)^{3/2}}\dx$
\item $\ds \int x\sqrt{1-x^4}\dx$

\end{enumerate}

\end{document}







