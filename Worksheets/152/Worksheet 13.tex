\documentclass[10pt]{article}

\usepackage{enumerate}
\usepackage{amsmath}
\usepackage{amssymb}
\usepackage{amsthm}
\usepackage{array}
\usepackage[all]{xy}
\usepackage{fancyhdr}
\usepackage{euscript}
\usepackage{graphics}
\usepackage{cancel}
\usepackage{fancybox}
\usepackage{tikz}
\usepackage{tikz-3dplot}
\usepackage{pgf}
\usepackage{pgfplots}
\usepackage[all]{xy}
\usepackage{graphicx}
\usepackage{hyperref}
\pgfplotsset{compat=1.14}

\usepackage{pstricks}
\usepackage{pst-plot}

\usepackage{setspace}
\onehalfspacing

\setlength{\oddsidemargin}{.5in}
\setlength{\evensidemargin}{.5in}
\setlength{\textwidth}{6.in}
\setlength{\topmargin}{0in}
\setlength{\headsep}{.20in}
\setlength{\textheight}{8.5in}


\pdfpagewidth 8.5in
 \pdfpageheight 11in


%General
\newcommand{\WW}{\mathbb {W}}
\newcommand{\ZZ}{\mathbb{Z}}
\newcommand{\RR}{\mathbb {R}}
\newcommand{\II}{\mathbb {I}}
\newcommand{\QQ}{\mathbb {Q}}
\newcommand{\CC}{\mathbf C}
\newcommand{\NN}{\mathbb {N}}
\newcommand{\Zn}[1]{\mathbf{Z}/#1\mathbf{Z}}
\newcommand{\Znx}[1]{(\mathbf{Z}/#1\mathbf{Z})^\times}
\newcommand{\X}{\times} 
\newcommand{\set}[2]{\left\{#1 : #2\right\}}          
\newcommand{\sett}[1]{\left\{#1\right\}}                
\newcommand{\nonempty}{\neq\varnothing}
\newcommand{\ds}{\displaystyle}
\newcommand{\abs}[1]{\left| {#1} \right|}
\newcommand{\qedbox}{\rule{2mm}{2mm}}
\renewcommand{\qedsymbol}{\qedbox}											
\newcommand{\aand}{\qquad\hbox{and}\qquad}
\newcommand{\e}{\varepsilon}
\newcommand{\tto}{\rightrightarrows}
\newcommand{\gs}{\geqslant}
\newcommand{\ls}{\leqslant}
\renewcommand{\tilde}{\widetilde}
\renewcommand{\hat}{\widehat}
\newcommand{\norm}[1]{\left\| #1 \right\|}
\newcommand{\md}[3]{#1\equiv#2\;(\mathrm{mod}\;#3)}     
\newcommand{\gen}[1]{\left\langle #1 \right\rangle}
\renewcommand{\Re}{\operatorname{Re}}
\renewcommand{\Im}{\operatorname{Im}}
\newcommand{\zero}{\boldsymbol{0}}

\newcommand{\be}[1]{\textbf{\emph{#1}}}
\newcommand{\hhat}[1]{\hat{\! \hat{#1}}}

\newcommand{\fto}[1]{\xrightarrow{\hspace{4pt} #1 \hspace{4pt}}}
\newcommand{\flto}[1]{\xrightarrow{\quad #1 \quad}}



\newcommand{\dist}{\operatorname{dist}}
\newcommand{\esssup}{\operatorname{ess\:sup}}
\newcommand{\id}{\operatorname{id}}
\newcommand{\card}{\operatorname{card}}

\newcommand{\dmu}{\:\mathrm{d}\mu}
\newcommand{\dm}{\:\mathrm{d}m}
\newcommand{\dx}{\:\mathrm{d}x}
\newcommand{\dt}{\:\mathrm{d}t}
\newcommand{\dz}{\:\mathrm{d}z}
\newcommand{\dtheta}{\:\mathrm{d}\theta}
\newcommand{\dw}{\:\mathrm{d}w}

%Algebra
\newcommand{\Sym}{\operatorname {Sym}}
\newcommand{\Stab}{\operatorname {Stab}}
\newcommand{\M}{\operatorname{M}}
\newcommand{\GL}{\operatorname{GL}}
\newcommand{\PGL}{\operatorname{PGL}}
\newcommand{\SL}{\operatorname{SL}}
\newcommand{\PSL}{\operatorname{PSL}}
\newcommand{\Heis}{\operatorname{Heis}}
\newcommand{\Aff}{\operatorname{Aff}}
\newcommand{\Aut}{\operatorname{Aut}}
\newcommand{\image}{\operatorname{im}}
\newcommand{\Syl}[2]{\operatorname{\emph{Syl}}_{#1}\left(#2\right)}
\newcommand{\Hom}{\operatorname{Hom}}
\newcommand{\Tor}{\operatorname{Tor}}
\newcommand{\Gal}{\operatorname{Gal}}
\newcommand{\ch}{\operatorname{ch}}
\newcommand{\rad}{\operatorname{rad}}
\newcommand{\iso}{\cong}
\newcommand{\normal}{\unlhd}
\newcommand{\semi}{\rtimes}
\newcommand{\Nm}{\operatorname {N}}
\newcommand{\Tr}{\operatorname {Tr}}
\newcommand{\disc}{\operatorname {disc}}








%Euler Script Characters
\newcommand{\esa}{\EuScript{A}}
\newcommand{\esb}{\EuScript{B}}
\newcommand{\esc}{\EuScript{C}}
\newcommand{\esd}{\EuScript{D}}
\newcommand{\ese}{\EuScript{E}}
\newcommand{\esf}{\EuScript{F}}
\newcommand{\esg}{\EuScript{G}}
\newcommand{\esh}{\EuScript{H}}
\newcommand{\esi}{\EuScript{I}}
\newcommand{\esj}{\EuScript{J}}
\newcommand{\esk}{\EuScript{K}}
\newcommand{\esl}{\EuScript{L}}
\newcommand{\esm}{\EuScript{M}}
\newcommand{\esn}{\EuScript{N}}
\newcommand{\eso}{\EuScript{O}}
\newcommand{\esp}{\EuScript{P}}
\newcommand{\esq}{\EuScript{Q}}
\newcommand{\esr}{\EuScript{R}}
\newcommand{\ess}{\EuScript{S}}
\newcommand{\est}{\EuScript{T}}
\newcommand{\esu}{\EuScript{U}}
\newcommand{\esv}{\EuScript{V}}
\newcommand{\esw}{\EuScript{W}}
\newcommand{\esx}{\EuScript{X}}
\newcommand{\esy}{\EuScript{Y}}
\newcommand{\esz}{\EuScript{Z}}

%Calligraphic Characters
\newcommand{\cala}{\mathcal{A}}
\newcommand{\calb}{\mathcal{B}}
\newcommand{\calc}{\mathcal{C}}
\newcommand{\cald}{\mathcal{D}}
\newcommand{\cale}{\mathcal{E}}
\newcommand{\calf}{\mathcal{F}}
\newcommand{\calg}{\mathcal{G}}
\newcommand{\calh}{\mathcal{H}}
\newcommand{\cali}{\mathcal{I}}
\newcommand{\calj}{\mathcal{J}}
\newcommand{\calk}{\mathcal{K}}
\newcommand{\call}{\mathcal{L}}
\newcommand{\calm}{\mathcal{M}}
\newcommand{\caln}{\mathcal{N}}
\newcommand{\calo}{\mathcal{O}}
\newcommand{\calp}{\mathcal{P}}
\newcommand{\calq}{\mathcal{Q}}
\newcommand{\calr}{\mathcal{R}}
\newcommand{\cals}{\mathcal{S}}
\newcommand{\calt}{\mathcal{T}}
\newcommand{\calu}{\mathcal{U}}
\newcommand{\calv}{\mathcal{V}}
\newcommand{\calw}{\mathcal{W}}
\newcommand{\calx}{\mathcal{X}}
\newcommand{\caly}{\mathcal{Y}}
\newcommand{\calz}{\mathcal{Z}}

%Gothic Characters
\newcommand{\fraka}{\mathfrak{a}}
\newcommand{\frakb}{\mathfrak{b}}
\newcommand{\frakc}{\mathfrak{c}}
\newcommand{\frakd}{\mathfrak{d}}
\newcommand{\frake}{\mathfrak{e}}
\newcommand{\frakf}{\mathfrak{f}}
\newcommand{\frakg}{\mathfrak{g}}
\newcommand{\frakh}{\mathfrak{h}}
\newcommand{\fraki}{\mathfrak{i}}
\newcommand{\frakj}{\mathfrak{j}}
\newcommand{\frakk}{\mathfrak{k}}
\newcommand{\frakl}{\mathfrak{l}}
\newcommand{\frakm}{\mathfrak{m}}
\newcommand{\frakn}{\mathfrak{n}}
\newcommand{\frako}{\mathfrak{o}}
\newcommand{\frakp}{\mathfrak{p}}
\newcommand{\frakq}{\mathfrak{q}}
\newcommand{\frakr}{\mathfrak{r}}
\newcommand{\fraks}{\mathfrak{s}}
\newcommand{\frakt}{\mathfrak{t}}
\newcommand{\fraku}{\mathfrak{u}}
\newcommand{\frakv}{\mathfrak{v}}
\newcommand{\frakw}{\mathfrak{w}}
\newcommand{\frakx}{\mathfrak{x}}
\newcommand{\fraky}{\mathfrak{y}}
\newcommand{\frakz}{\mathfrak{z}}

\newcommand{\frakA}{\mathfrak{A}}
\newcommand{\frakB}{\mathfrak{B}}
\newcommand{\frakC}{\mathfrak{C}}
\newcommand{\frakD}{\mathfrak{D}}
\newcommand{\frakE}{\mathfrak{E}}
\newcommand{\frakF}{\mathfrak{F}}
\newcommand{\frakG}{\mathfrak{G}}
\newcommand{\frakH}{\mathfrak{H}}
\newcommand{\frakI}{\mathfrak{I}}
\newcommand{\frakJ}{\mathfrak{J}}
\newcommand{\frakK}{\mathfrak{K}}
\newcommand{\frakL}{\mathfrak{L}}
\newcommand{\frakM}{\mathfrak{M}}
\newcommand{\frakN}{\mathfrak{N}}
\newcommand{\frakO}{\mathfrak{O}}
\newcommand{\frakP}{\mathfrak{P}}
\newcommand{\frakQ}{\mathfrak{Q}}
\newcommand{\frakR}{\mathfrak{R}}
\newcommand{\frakS}{\mathfrak{S}}
\newcommand{\frakT}{\mathfrak{T}}
\newcommand{\frakU}{\mathfrak{U}}
\newcommand{\frakV}{\mathfrak{V}}
\newcommand{\frakW}{\mathfrak{W}}
\newcommand{\frakX}{\mathfrak{X}}
\newcommand{\frakY}{\mathfrak{Y}}
\newcommand{\frakZ}{\mathfrak{Z}}

%Lowercase Bold Letters
\newcommand{\bfa}{\mathbf{a}}
\newcommand{\bfb}{\mathbf{b}}
\newcommand{\bfc}{\mathbf{c}}
\newcommand{\bfd}{\mathbf{d}}
\newcommand{\bfe}{\mathbf{e}}
\newcommand{\bff}{\mathbf{f}}
\newcommand{\bfg}{\mathbf{g}}
\newcommand{\bfh}{\mathbf{h}}
\newcommand{\bfi}{\mathbf{i}}
\newcommand{\bfj}{\mathbf{j}}
\newcommand{\bfk}{\mathbf{k}}
\newcommand{\bfl}{\mathbf{l}}
\newcommand{\bfm}{\mathbf{m}}
\newcommand{\bfn}{\mathbf{n}}
\newcommand{\bfo}{\mathbf{o}}
\newcommand{\bfp}{\mathbf{p}}
\newcommand{\bfq}{\mathbf{q}}
\newcommand{\bfr}{\mathbf{r}}
\newcommand{\bfs}{\mathbf{s}}
\newcommand{\bft}{\mathbf{t}}
\newcommand{\bfu}{\mathbf{u}}
\newcommand{\bfv}{\mathbf{v}}
\newcommand{\bfw}{\mathbf{w}}
\newcommand{\bfx}{\mathbf{x}}
\newcommand{\bfy}{\mathbf{y}}
\newcommand{\bfz}{\mathbf{z}}




%Customized Theorem Environments
\newtheoremstyle%
{custom}%
{}%                         Space above
{}%													Space below
{}%													Body font
{}%                         Indent amount
{}%                         Theorem head font
{.}%                        Punctuation after heading
{ }%                        Space after heading
{\thmname{}%                Additional specifications for theorem head
\thmnumber{}%
\thmnote{\bfseries #3}}%

\newtheoremstyle%
{Theorem}%
{}%
{}%
{\itshape}%
{}%
{}%
{.}%
{ }%
{\thmname{\bfseries #1}%
\thmnumber{\;\bfseries #2}%
\thmnote{\;(\bfseries #3)}}%

%Theorem Environments
\theoremstyle{Theorem}
\newtheorem{theorem}{Theorem}[section]
\newtheorem{cor}{Corollary}[section]
\newtheorem{lemma}{Lemma}[section]
\newtheorem{prop}{Proposition}[section]
\newtheorem*{nonumthm}{Theorem}
\newtheorem*{nonumprop}{Proposition}
\theoremstyle{definition}
\newtheorem{definition}{Definition}[section]
\newtheorem*{answer}{Answer}
\newtheorem*{solution}{Solution}
\newtheorem*{nonumdfn}{Definition}
\newtheorem*{nonumex}{Example}
\newtheorem{ex}{Example}[section]
\theoremstyle{remark}
\newtheorem{remark}{Remark}[section]
\newtheorem*{note}{Note}
\newtheorem*{notation}{Notation}
\theoremstyle{custom}
\newtheorem*{cust}{Definition}
\fancypagestyle{firststyle}
{
  % \fancyhead[L]{\textbf{Name:}}
   \fancyhead[R]{\textbf{Worksheet 13: Polar Curves and Coordinates, Areas and Lengths}}
   \fancyfoot[R]{ Thomas Luckner } %{\footnotesize Page \thepage\ of \pageref{LastPage}}
}



\begin{document}
\thispagestyle{firststyle}
\pagestyle{plain}


\noindent Thoughts: \\
This is the last topic for you guys! I know I know this is great news. Let's finish strong.  First we have to explain a polar coordinate. Let's consider the point $\left( \dfrac{\sqrt{2}}{2},\dfrac{\sqrt{2}}{2}\right)$. This point is on the unit circle with a an angle of $\pi/4$ with respect to the $x$-axis. Thus, its polar coordinates are (I know you do not know what that means yet, but this shoulld help you take a guess) $(1, \pi/4)$.  Any guesses? The idea behind polar coordinates is the following:
\begin{definition}[Polar Coordinate]
The point $(x,y)$ (a Cartesian point) has the polar form $(r,\theta)$ where $r$ is the distance of the point $(x,y)$ from the origin and $\theta$ is the angle with respect to the $x$-axis. 
\end{definition}
Let's do a general example: $(x,y)$.  The best way to illustrate the conversion to polar coordinates is by a picture.
\begin{center}
  \begin{tikzpicture}

      \draw[thick,->] (0,0) -- (5,0) node[right] {};
      \draw[thick,->] (0,0) -- (0,5) node[above] {};
       \draw[thick, blue, -] (0,0) -- (3,3) node[left] {};
       \draw[thick, blue, -] (0,0) -- (3,0) node[left] {};
       \draw[thick, blue, -] (3,0) -- (3,3) node[left] {};
       \draw (3,3) node[above] {$(x,y)$};
       \draw (1.5, 1.5) node[above left] {$r$};
       \draw (1.5,0) node[below] {$x$};
       \draw (3, 1.5) node[right] {$y$};
       \draw (.5, 0) node [above] {$\theta$};
    \end{tikzpicture}
\end{center}
This image should work wonders! How do we find $r$? That's right! Pythagorean Theorem! Thus, $r^2=x^2+y^2$ is quite useful. How do we find $\theta$? Soh Cah Toa! Any of them! For description sake I will write all of them: $\sin\theta=\dfrac{y}{r}$, $\cos\theta=\dfrac{x}{r}$, and $\tan\theta = \dfrac{y}{x}$. In the case of going from Cartesian to polar, you will see all of these work as long as you find $r$ first (tangent you do not need $r$). Going the other direction (polar to Cartesian) is a different story! Knowing $r$ and $\theta$ to find $x$ and $y$ we can do some manipulation and the useful equations are: $x=r\cos\theta$ and $y=r\sin\theta$. That's all you need to know to convert between polar and Cartesian coordinates!\\
With new coordinates comes new equations. Let's see the general form: $r=f(\theta)$ or $f(r,\theta)=0$. An example!
\begin{ex}
$r=2\cos(\theta)$. Try sketching this curve! I will show how to convert to Cartesian equations! 
\[
r=2\left(\dfrac{x}{r}\right) \Rightarrow r^2=2x \Rightarrow x^2+y^2=2x \Rightarrow (x-1)^2+y^2=1
\]
The last part is by completing the square. Hopefully you see that this is the equation of a circle centered at (1,0) with radius 1.  Another similar example is the equation $r=1+\sin\theta$ which is the equation of a cardioid.
\end{ex}
Now let's do some calculus! 
\[
\dfrac{dy}{dx}=\dfrac{\dfrac{dy}{d\theta}}{\dfrac{dx}{d\theta}}=\dfrac{\dfrac{d(r\sin\theta)}{d\theta}}{\dfrac{d(r\cos\theta)}{d\theta}}=\dfrac{\sin\theta \cdot\dfrac{dr}{d\theta}+\cos\theta \cdot r }{\cos\theta \cdot \dfrac{dr}{d\theta}-\sin\theta \cdot r}
\]
These are all the different ways you can find $\dfrac{dy}{dx}$ with polar equations! Thus, now applications like tangent lines are accessible.\\
We can also use integration to tell us things. First, we can determine the area of a sector of the function when we have polar equations. What is the sector of a polar equation you ask? Well, the sector is the area of the region made my drawing a straight line from each point $(a,f(a))$ and $(b,f(b))$ where $a$ and $b$ are the end points to the origin. Then use the function $r=f(\theta)$ to be the last side of the solid. This shape has the following formula for area:
\[
A=\int_a^b (1/2)(f(\theta))^2 d\theta=\int_a^b(1/2)r^2d\theta.
\]
Now we have the idea of arc length again. I do not want to go into why the formula is the way it is. I will instead leave it up to you to ask me if you want to know where it comes from.  Here is the formula. 
\[
L=\int_a^b \sqrt{r^2+\left(\dfrac{dr}{d\theta}\right)^2}d\theta
\]
These last two concepts are more plug and chug than anything else. That concludes everything for the class! 
\newpage
\noindent Problems:
\begin{enumerate}[1.]
\item Convert the following from Cartesian to Polar for $r>0$ and $0\leq \theta <2\pi$.
\begin{enumerate}[a.]
\item $(-1,\sqrt{3})$
\item $(3\sqrt{3}, 3)$
\item $(1,-2)$
\end{enumerate}
\item Convert from Polar to Cartesian.
\begin{enumerate}[a.]
\item $(1,\pi)$
\item $(-2, 3\pi/4)$
\item $(2, 7\pi/6)$
\end{enumerate}
\item Convert the Cartesian equation to polar.
\begin{enumerate}[a.]
\item $y=2$
\item $xy=4$
\item $x^2+y^2=2x$
\end{enumerate}
\item Convert the polar equations to Cartesian.
\begin{enumerate}[a.]
\item $r=4\sin(3\theta)$
\item $r^2=9\sin(2\theta)$
\item $r=1-2\sin\theta$
\item $r=3+4\cos\theta$
\end{enumerate}
\item Find the slope of the tangent lines for the following polar equations.
\begin{enumerate}[a.]
\item $r=1/\theta$
\item $r=1+2\cos\theta$
\item $r=e^{\theta}$
\end{enumerate}
\item Find the area inside $r=\sqrt{3}\cos\theta$ and inside $r=\sin\theta$.
\item Find the length of $r=2(1+\cos\theta$
\item Find the area enclosed by the curve $r=1+5\sin\theta$.
\end{enumerate}
\end{document}







