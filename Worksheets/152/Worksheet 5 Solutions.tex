\documentclass[10pt]{article}

\usepackage{enumerate}
\usepackage{amsmath}
\usepackage{amssymb}
\usepackage{amsthm}
\usepackage{array}
\usepackage[all]{xy}
\usepackage{fancyhdr}
\usepackage{euscript}
\usepackage{graphics}
\usepackage{cancel}
\usepackage{fancybox}
\usepackage{tikz}
\usepackage{tikz-3dplot}
\usepackage{pgf}
\usepackage{pgfplots}
\usepackage[all]{xy}
\usepackage{graphicx}
\usepackage{hyperref}
\pgfplotsset{compat=1.14}

\usepackage{pstricks}
\usepackage{pst-plot}

\usepackage{setspace}
\onehalfspacing

\setlength{\oddsidemargin}{.5in}
\setlength{\evensidemargin}{.5in}
\setlength{\textwidth}{6.in}
\setlength{\topmargin}{0in}
\setlength{\headsep}{.20in}
\setlength{\textheight}{8.5in}


\pdfpagewidth 8.5in
 \pdfpageheight 11in


%General
\newcommand{\WW}{\mathbb {W}}
\newcommand{\ZZ}{\mathbb{Z}}
\newcommand{\RR}{\mathbb {R}}
\newcommand{\II}{\mathbb {I}}
\newcommand{\QQ}{\mathbb {Q}}
\newcommand{\CC}{\mathbf C}
\newcommand{\NN}{\mathbb {N}}
\newcommand{\Zn}[1]{\mathbf{Z}/#1\mathbf{Z}}
\newcommand{\Znx}[1]{(\mathbf{Z}/#1\mathbf{Z})^\times}
\newcommand{\X}{\times} 
\newcommand{\set}[2]{\left\{#1 : #2\right\}}          
\newcommand{\sett}[1]{\left\{#1\right\}}                
\newcommand{\nonempty}{\neq\varnothing}
\newcommand{\ds}{\displaystyle}
\newcommand{\abs}[1]{\left| {#1} \right|}
\newcommand{\qedbox}{\rule{2mm}{2mm}}
\renewcommand{\qedsymbol}{\qedbox}											
\newcommand{\aand}{\qquad\hbox{and}\qquad}
\newcommand{\e}{\varepsilon}
\newcommand{\tto}{\rightrightarrows}
\newcommand{\gs}{\geqslant}
\newcommand{\ls}{\leqslant}
\renewcommand{\tilde}{\widetilde}
\renewcommand{\hat}{\widehat}
\newcommand{\norm}[1]{\left\| #1 \right\|}
\newcommand{\md}[3]{#1\equiv#2\;(\mathrm{mod}\;#3)}     
\newcommand{\gen}[1]{\left\langle #1 \right\rangle}
\renewcommand{\Re}{\operatorname{Re}}
\renewcommand{\Im}{\operatorname{Im}}
\newcommand{\zero}{\boldsymbol{0}}

\newcommand{\be}[1]{\textbf{\emph{#1}}}
\newcommand{\hhat}[1]{\hat{\! \hat{#1}}}

\newcommand{\fto}[1]{\xrightarrow{\hspace{4pt} #1 \hspace{4pt}}}
\newcommand{\flto}[1]{\xrightarrow{\quad #1 \quad}}



\newcommand{\dist}{\operatorname{dist}}
\newcommand{\esssup}{\operatorname{ess\:sup}}
\newcommand{\id}{\operatorname{id}}
\newcommand{\card}{\operatorname{card}}

\newcommand{\dmu}{\:\mathrm{d}\mu}
\newcommand{\dm}{\:\mathrm{d}m}
\newcommand{\dx}{\:\mathrm{d}x}
\newcommand{\dt}{\:\mathrm{d}t}
\newcommand{\dz}{\:\mathrm{d}z}
\newcommand{\dtheta}{\:\mathrm{d}\theta}
\newcommand{\dw}{\:\mathrm{d}w}

%Algebra
\newcommand{\Sym}{\operatorname {Sym}}
\newcommand{\Stab}{\operatorname {Stab}}
\newcommand{\M}{\operatorname{M}}
\newcommand{\GL}{\operatorname{GL}}
\newcommand{\PGL}{\operatorname{PGL}}
\newcommand{\SL}{\operatorname{SL}}
\newcommand{\PSL}{\operatorname{PSL}}
\newcommand{\Heis}{\operatorname{Heis}}
\newcommand{\Aff}{\operatorname{Aff}}
\newcommand{\Aut}{\operatorname{Aut}}
\newcommand{\image}{\operatorname{im}}
\newcommand{\Syl}[2]{\operatorname{\emph{Syl}}_{#1}\left(#2\right)}
\newcommand{\Hom}{\operatorname{Hom}}
\newcommand{\Tor}{\operatorname{Tor}}
\newcommand{\Gal}{\operatorname{Gal}}
\newcommand{\ch}{\operatorname{ch}}
\newcommand{\rad}{\operatorname{rad}}
\newcommand{\iso}{\cong}
\newcommand{\normal}{\unlhd}
\newcommand{\semi}{\rtimes}
\newcommand{\Nm}{\operatorname {N}}
\newcommand{\Tr}{\operatorname {Tr}}
\newcommand{\disc}{\operatorname {disc}}








%Euler Script Characters
\newcommand{\esa}{\EuScript{A}}
\newcommand{\esb}{\EuScript{B}}
\newcommand{\esc}{\EuScript{C}}
\newcommand{\esd}{\EuScript{D}}
\newcommand{\ese}{\EuScript{E}}
\newcommand{\esf}{\EuScript{F}}
\newcommand{\esg}{\EuScript{G}}
\newcommand{\esh}{\EuScript{H}}
\newcommand{\esi}{\EuScript{I}}
\newcommand{\esj}{\EuScript{J}}
\newcommand{\esk}{\EuScript{K}}
\newcommand{\esl}{\EuScript{L}}
\newcommand{\esm}{\EuScript{M}}
\newcommand{\esn}{\EuScript{N}}
\newcommand{\eso}{\EuScript{O}}
\newcommand{\esp}{\EuScript{P}}
\newcommand{\esq}{\EuScript{Q}}
\newcommand{\esr}{\EuScript{R}}
\newcommand{\ess}{\EuScript{S}}
\newcommand{\est}{\EuScript{T}}
\newcommand{\esu}{\EuScript{U}}
\newcommand{\esv}{\EuScript{V}}
\newcommand{\esw}{\EuScript{W}}
\newcommand{\esx}{\EuScript{X}}
\newcommand{\esy}{\EuScript{Y}}
\newcommand{\esz}{\EuScript{Z}}

%Calligraphic Characters
\newcommand{\cala}{\mathcal{A}}
\newcommand{\calb}{\mathcal{B}}
\newcommand{\calc}{\mathcal{C}}
\newcommand{\cald}{\mathcal{D}}
\newcommand{\cale}{\mathcal{E}}
\newcommand{\calf}{\mathcal{F}}
\newcommand{\calg}{\mathcal{G}}
\newcommand{\calh}{\mathcal{H}}
\newcommand{\cali}{\mathcal{I}}
\newcommand{\calj}{\mathcal{J}}
\newcommand{\calk}{\mathcal{K}}
\newcommand{\call}{\mathcal{L}}
\newcommand{\calm}{\mathcal{M}}
\newcommand{\caln}{\mathcal{N}}
\newcommand{\calo}{\mathcal{O}}
\newcommand{\calp}{\mathcal{P}}
\newcommand{\calq}{\mathcal{Q}}
\newcommand{\calr}{\mathcal{R}}
\newcommand{\cals}{\mathcal{S}}
\newcommand{\calt}{\mathcal{T}}
\newcommand{\calu}{\mathcal{U}}
\newcommand{\calv}{\mathcal{V}}
\newcommand{\calw}{\mathcal{W}}
\newcommand{\calx}{\mathcal{X}}
\newcommand{\caly}{\mathcal{Y}}
\newcommand{\calz}{\mathcal{Z}}

%Gothic Characters
\newcommand{\fraka}{\mathfrak{a}}
\newcommand{\frakb}{\mathfrak{b}}
\newcommand{\frakc}{\mathfrak{c}}
\newcommand{\frakd}{\mathfrak{d}}
\newcommand{\frake}{\mathfrak{e}}
\newcommand{\frakf}{\mathfrak{f}}
\newcommand{\frakg}{\mathfrak{g}}
\newcommand{\frakh}{\mathfrak{h}}
\newcommand{\fraki}{\mathfrak{i}}
\newcommand{\frakj}{\mathfrak{j}}
\newcommand{\frakk}{\mathfrak{k}}
\newcommand{\frakl}{\mathfrak{l}}
\newcommand{\frakm}{\mathfrak{m}}
\newcommand{\frakn}{\mathfrak{n}}
\newcommand{\frako}{\mathfrak{o}}
\newcommand{\frakp}{\mathfrak{p}}
\newcommand{\frakq}{\mathfrak{q}}
\newcommand{\frakr}{\mathfrak{r}}
\newcommand{\fraks}{\mathfrak{s}}
\newcommand{\frakt}{\mathfrak{t}}
\newcommand{\fraku}{\mathfrak{u}}
\newcommand{\frakv}{\mathfrak{v}}
\newcommand{\frakw}{\mathfrak{w}}
\newcommand{\frakx}{\mathfrak{x}}
\newcommand{\fraky}{\mathfrak{y}}
\newcommand{\frakz}{\mathfrak{z}}

\newcommand{\frakA}{\mathfrak{A}}
\newcommand{\frakB}{\mathfrak{B}}
\newcommand{\frakC}{\mathfrak{C}}
\newcommand{\frakD}{\mathfrak{D}}
\newcommand{\frakE}{\mathfrak{E}}
\newcommand{\frakF}{\mathfrak{F}}
\newcommand{\frakG}{\mathfrak{G}}
\newcommand{\frakH}{\mathfrak{H}}
\newcommand{\frakI}{\mathfrak{I}}
\newcommand{\frakJ}{\mathfrak{J}}
\newcommand{\frakK}{\mathfrak{K}}
\newcommand{\frakL}{\mathfrak{L}}
\newcommand{\frakM}{\mathfrak{M}}
\newcommand{\frakN}{\mathfrak{N}}
\newcommand{\frakO}{\mathfrak{O}}
\newcommand{\frakP}{\mathfrak{P}}
\newcommand{\frakQ}{\mathfrak{Q}}
\newcommand{\frakR}{\mathfrak{R}}
\newcommand{\frakS}{\mathfrak{S}}
\newcommand{\frakT}{\mathfrak{T}}
\newcommand{\frakU}{\mathfrak{U}}
\newcommand{\frakV}{\mathfrak{V}}
\newcommand{\frakW}{\mathfrak{W}}
\newcommand{\frakX}{\mathfrak{X}}
\newcommand{\frakY}{\mathfrak{Y}}
\newcommand{\frakZ}{\mathfrak{Z}}

%Lowercase Bold Letters
\newcommand{\bfa}{\mathbf{a}}
\newcommand{\bfb}{\mathbf{b}}
\newcommand{\bfc}{\mathbf{c}}
\newcommand{\bfd}{\mathbf{d}}
\newcommand{\bfe}{\mathbf{e}}
\newcommand{\bff}{\mathbf{f}}
\newcommand{\bfg}{\mathbf{g}}
\newcommand{\bfh}{\mathbf{h}}
\newcommand{\bfi}{\mathbf{i}}
\newcommand{\bfj}{\mathbf{j}}
\newcommand{\bfk}{\mathbf{k}}
\newcommand{\bfl}{\mathbf{l}}
\newcommand{\bfm}{\mathbf{m}}
\newcommand{\bfn}{\mathbf{n}}
\newcommand{\bfo}{\mathbf{o}}
\newcommand{\bfp}{\mathbf{p}}
\newcommand{\bfq}{\mathbf{q}}
\newcommand{\bfr}{\mathbf{r}}
\newcommand{\bfs}{\mathbf{s}}
\newcommand{\bft}{\mathbf{t}}
\newcommand{\bfu}{\mathbf{u}}
\newcommand{\bfv}{\mathbf{v}}
\newcommand{\bfw}{\mathbf{w}}
\newcommand{\bfx}{\mathbf{x}}
\newcommand{\bfy}{\mathbf{y}}
\newcommand{\bfz}{\mathbf{z}}




%Customized Theorem Environments
\newtheoremstyle%
{custom}%
{}%                         Space above
{}%													Space below
{}%													Body font
{}%                         Indent amount
{}%                         Theorem head font
{.}%                        Punctuation after heading
{ }%                        Space after heading
{\thmname{}%                Additional specifications for theorem head
\thmnumber{}%
\thmnote{\bfseries #3}}%

\newtheoremstyle%
{Theorem}%
{}%
{}%
{\itshape}%
{}%
{}%
{.}%
{ }%
{\thmname{\bfseries #1}%
\thmnumber{\;\bfseries #2}%
\thmnote{\;(\bfseries #3)}}%

%Theorem Environments
\theoremstyle{Theorem}
\newtheorem{theorem}{Theorem}[section]
\newtheorem{cor}{Corollary}[section]
\newtheorem{lemma}{Lemma}[section]
\newtheorem{prop}{Proposition}[section]
\newtheorem*{nonumthm}{Theorem}
\newtheorem*{nonumprop}{Proposition}
\theoremstyle{definition}
\newtheorem{definition}{Definition}[section]
\newtheorem*{answer}{Answer}
\newtheorem*{solution}{Solution}
\newtheorem*{nonumdfn}{Definition}
\newtheorem*{nonumex}{Example}
\newtheorem{ex}{Example}[section]
\theoremstyle{remark}
\newtheorem{remark}{Remark}[section]
\newtheorem*{note}{Note}
\newtheorem*{notation}{Notation}
\theoremstyle{custom}
\newtheorem*{cust}{Definition}
\fancypagestyle{firststyle}
{
   \fancyhead[L]{\textbf{Name:}}
   \fancyhead[R]{\textbf{Worksheet 5: Series- Telescoping, Geometric, and 0-test}}
   \fancyfoot[R]{ Thomas Luckner } %{\footnotesize Page \thepage\ of \pageref{LastPage}}
}



\begin{document}
\thispagestyle{firststyle}
\pagestyle{plain}


\noindent Thoughts: \\
Last worksheet I left you with 2 examples to think about. Here they are:
\[
\sum_{n=1}^{\infty}\dfrac{1}{n}-\dfrac{1}{n+1}
\]
\[
\sum_{n=0}^{\infty}\dfrac{2}{3^n}.
\]
For now, I want to put these on the back burner.  Let's instead start with an idea I hinted at before in the previous worksheet with a sequence with all values greater than or equal to 1.  We noticed this would diverge. Let's try to strengthen this statement. What if I had the following infinite series:
\[
\sum_{n=1}^{\infty}\dfrac{1}{2}.
\]
Does this converge. The answer is no. If you add a number to itself an infinite amount of times it must be infinite! Let's try to look at another example.
\[
\sum_{n=1}^{\infty}\sin\left(\dfrac{n\pi}{2}\right)
\]
This is a little more difficult to see, but notice that 1 will appear an infinite amount of time, $-1$ will appear an infinite amount of time, and 0 will appear an infinite amount of time. This feels weird. Your instinct is to say well then it is 0. This is divergent since we have no way of canceling 1's and $-1$'s out perfectly! Thus, divergent. There is a small pattern here though. Notice the limit of the sequences are 0! Thus, we have determined a conjecture.
\begin{theorem}[Divergence Test (0-test)]
If the sequence $a_n$ does not converge to 0 then 
\[
\sum_{n=1}^{\infty}a_n
\]
is divergent. This includes any starting place ($n=0$, $n=2$, etc.).  
\end{theorem}
Note this says nothing about convergence! For example,
\[
\sum_{n=1}^{\infty}\dfrac{1}{n}
\]
is divergent, but the sequence, $\dfrac{1}{n}$ (harmonic series), converges to 0. The reason this series is divergent will soon come, but for now just assume this to be divergent for the sake of making this point of the 0-test. Thus, the above theorem is all we can say about series! If you'd like a more formal proof of this, let me know! I can write up a nice one for you to get the point across.
\newpage
Telescoping and Geometric Series:\\\\
Now we can bring up those examples again. For both of them we will consider what is called the $n$-th partial sum. Let's start with the first. 
\[
\sum_{n=1}^{\infty}\dfrac{1}{n}-\dfrac{1}{n+1}
\]
The first partial sum is just the first term.
\[
S_1=\dfrac{1}{1}-\dfrac{1}{2}=1-\dfrac{1}{2}
\]
Now the second partial sum adds on the next term.
\[
S_2=S_1+\dfrac{1}{2}-\dfrac{1}{3}=1-\dfrac{1}{2}+\dfrac{1}{2}-\dfrac{1}{3}=1-\dfrac{1}{3}
\]
Notice the middle terms cancel each other! Let's do one more to get the point across.
\[
S_3=S_2+\dfrac{1}{3}-\dfrac{1}{4}=1-\dfrac{1}{3}+\dfrac{1}{3}-\dfrac{1}{4}=1-\dfrac{1}{4}
\]
Now we can generalize the pattern for some $n$.
\[
S_n=S_{n-1}+\dfrac{1}{n}-\dfrac{1}{n+1}=1-\dfrac{1}{n}+\dfrac{1}{n}-\dfrac{1}{n+1}=1-\dfrac{1}{n+1}
\]
Let's zero in on what $S_n$ is in terms of sigma notation.
\[
S_n=\sum_{k=1}^{n}\dfrac{1}{k}-\dfrac{1}{k+1}
\]
Here we change to $k$ to use the $n$ we mention in the partial sum. Thus, we have the following relationship between the original sum and the partial sum:
\[
\sum_{k=1}^{\infty}\dfrac{1}{k}-\dfrac{1}{k+1}=\lim_{n\rightarrow \infty}S_n.
\]
Thus, what we have left to do is take the limit of our $S_n$. In this case, $\ds \lim_{n\rightarrow \infty} S_n=1-0=1$.\\
A series that continually "eats itself" like this is called a telescoping series.\\
NOTE: They are not always obvious in that a partial fraction decomposition can expose one if you are noticing a similar trend when you start looking at partial sums!!\\\\
\newpage
Now let's consider the next example.
\[
\sum_{n=0}^{\infty}\dfrac{2}{3^n}.
\]
The sequence converges to 0, so our divergence test is no match for this. Let's look at $n$-th partial sum of this in an arbitrary form: $a=2$ and $r=\dfrac{1}{3}$.
\[
S_n=a+ar+ar^2+\cdots +ar^{n-1}+ar^n
\]
Now I want you to bare with me here since we are going to try to show the sum formula. Let's consider the $n$-th partial sum multiplied by $r$.
\[
rS_n=ar+ar^2+\cdots +ar^{n+1}
\]
Remember our goal is like telescoping where we want a formula for the $n$-th partial sum so we can take a limit. Let's take the difference and see what happens.
\[
S_n-rS_n=S_n(1-r)=a-ar^{n+1}=a(1-r^{n+1})\\
\Rightarrow S_n=\dfrac{a(1-r^{n+1})}{1-r}
\]
We have our formula, although out of left field. That's ok though1 Now we know where it comes from. Notice our infinite series is the following:
\[
\lim_{n\rightarrow \infty} S_n=\lim_{n\rightarrow \infty}\dfrac{a(1-r^{n+1})}{1-r}=\dfrac{a}{1-r}
\]
ONLY IF $|r|<1$ since the limit of $r^{n+1}$ goes to 0 only if this condition holds true! 
The formulas above for $S_n$ and the limit of $S_n$ are the formulas for the finite geometric series and infinite geometric series! The general form is as follows:
\[
\sum_{n=0}^{\infty}ar^n.
\]
Only convergent going to infinity if the condition on $r$ holds.\\
What is really nice about infinite geometric series is that the convergence or divergence of the series with the form above is completely determined by the condition on $r$! Thus, we have the following theorem.
\begin{theorem}[Geometric Series Test]
If a series is of infinite geometric form (see above) then\\
(1) if $|r|<1$, the series is convergent or,\\
(2) it is divergent ($|r|\geq 1$).
\end{theorem}
NOTE: Geometric series do not have to start at 0! You can start at 1 and have the following:
\[
\sum_{n=1}^{\infty}ar^n=\left(\sum_{n=0}^{\infty}ar^n\right)-a.
\]
\newpage
\noindent Problems: Determine if the series is convergent or divergent and give your reasoning. If geometric, find the sum!
\begin{enumerate}[1.]
\item $\ds \sum_{n=1}^{\infty} \dfrac{n-1}{3n-1}$
\begin{solution}
Divergence test or zero test says divergent!
\end{solution}
\item $\ds \sum_{n=0}^{\infty} \left(\dfrac{\pi}{3}\right)^n$
\begin{solution}
$\dfrac{\pi}{3}>1$ Thus, divergent geometric series!
\end{solution}
\item $\ds \sum_{n=1}^{\infty} \dfrac{3}{n(n+3)}$
\begin{solution}
Let's use partial fraction here to make a telescoping series.
\[
\dfrac{3}{n(n+3)}= \dfrac{A}{n}+\dfrac{B}{n+3}
\]
\[
\Rightarrow 3=A(n+3)+Bn
\]
\[
\Rightarrow A=1, B=-1.
\]
Thus, 
\[
\ds \sum_{n=1}^{\infty} \dfrac{3}{n(n+3)}=\ds \sum_{n=1}^{\infty} \dfrac{1}{n}-\dfrac{1}{n+3}
\]
Let's write a few partial sums.
\[
S_1=1-\dfrac{1}{4}
\]
\[
S_2=1-\dfrac{1}{4}+\dfrac{1}{2}-\dfrac{1}{5}
\]
\[
S_3=1-\dfrac{1}{4}+\dfrac{1}{2}-\dfrac{1}{5}+\dfrac{1}{3}-\dfrac{1}{6}
\]
\[
S_4=1-\dfrac{1}{4}+\dfrac{1}{2}-\dfrac{1}{5}+\dfrac{1}{3}-\dfrac{1}{6}+\dfrac{1}{4}-\dfrac{1}{7}=1+\dfrac{1}{2}-\dfrac{1}{5}+\dfrac{1}{3}-\dfrac{1}{6}-\dfrac{1}{7}
\]
\[
\cdots S_n=1+\dfrac{1}{2}+\dfrac{1}{3}-\dfrac{1}{n+1}-\dfrac{1}{n+2}-\dfrac{1}{n+3}
\]
Thus, 
\[
\ds \sum_{n=1}^{\infty} \dfrac{3}{n(n+3)}=1+\dfrac{1}{2}+\dfrac{1}{3}=\dfrac{11}{6}.
\]
\end{solution}
\newpage
\item $\ds \sum_{n=1}^{\infty}  \cos\left(\dfrac{1}{n^2}\right)-\cos\left(\dfrac{1}{(n+1)^2}\right)$
\begin{solution}
Let's find the $n$-th partial sum.
\[
S_1=\cos(1)-\cos(1/4)
\]
\[
S_2=\cos(1)-\cos(1/4)+\cos(1/4)-\cos(1/9)=\cos(1)-\cos(1/9)
\]
\[
\cdots S_n=\cos(1)-\cos(1/(n+1)^2)
\]
Thus, 
\[
\ds \sum_{n=1}^{\infty}  \cos\left(\dfrac{1}{n^2}\right)-\cos\left(\dfrac{1}{(n+1)^2}\right)=\cos(1)-\cos(0)=\cos(1)-1
\]

\end{solution}
\item $\ds \sum_{n=0}^{\infty} \left( \dfrac{1}{\sqrt{2}}\right)^n$
\begin{solution}
This is a converging geometric series since $\sqrt{2}>1$. Let's use the formula calling the series $s$.
\[
s=\dfrac{1}{1-\dfrac{1}{\sqrt{2}}}=\dfrac{2}{2-\sqrt{2}}
\]
\end{solution}
\item $\ds \sum_{n=1}^{\infty} 6(.9)^{n-1}$
\begin{solution}
Tricky geometric series! We can rewrite this as follows:
\[
\ds \sum_{n=1}^{\infty} 6(.9)^{n-1}=\ds \sum_{n=0}^{\infty}6(.9)^n.
\]
Now, since $.9<1$, we can find the convergent sum of the geometric series.
\[
s=\dfrac{6}{1-.9}=\dfrac{6}{.1}=60.
\]
\end{solution}
\end{enumerate}
\end{document}







