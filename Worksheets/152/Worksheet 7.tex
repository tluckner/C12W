\documentclass[10pt]{article}

\usepackage{enumerate}
\usepackage{amsmath}
\usepackage{amssymb}
\usepackage{amsthm}
\usepackage{array}
\usepackage[all]{xy}
\usepackage{fancyhdr}
\usepackage{euscript}
\usepackage{graphics}
\usepackage{cancel}
\usepackage{fancybox}
\usepackage{tikz}
\usepackage{tikz-3dplot}
\usepackage{pgf}
\usepackage{pgfplots}
\usepackage[all]{xy}
\usepackage{graphicx}
\usepackage{hyperref}
\pgfplotsset{compat=1.14}

\usepackage{pstricks}
\usepackage{pst-plot}

\usepackage{setspace}
\onehalfspacing

\setlength{\oddsidemargin}{.5in}
\setlength{\evensidemargin}{.5in}
\setlength{\textwidth}{6.in}
\setlength{\topmargin}{0in}
\setlength{\headsep}{.20in}
\setlength{\textheight}{8.5in}


\pdfpagewidth 8.5in
 \pdfpageheight 11in


%General
\newcommand{\WW}{\mathbb {W}}
\newcommand{\ZZ}{\mathbb{Z}}
\newcommand{\RR}{\mathbb {R}}
\newcommand{\II}{\mathbb {I}}
\newcommand{\QQ}{\mathbb {Q}}
\newcommand{\CC}{\mathbf C}
\newcommand{\NN}{\mathbb {N}}
\newcommand{\Zn}[1]{\mathbf{Z}/#1\mathbf{Z}}
\newcommand{\Znx}[1]{(\mathbf{Z}/#1\mathbf{Z})^\times}
\newcommand{\X}{\times} 
\newcommand{\set}[2]{\left\{#1 : #2\right\}}          
\newcommand{\sett}[1]{\left\{#1\right\}}                
\newcommand{\nonempty}{\neq\varnothing}
\newcommand{\ds}{\displaystyle}
\newcommand{\abs}[1]{\left| {#1} \right|}
\newcommand{\qedbox}{\rule{2mm}{2mm}}
\renewcommand{\qedsymbol}{\qedbox}											
\newcommand{\aand}{\qquad\hbox{and}\qquad}
\newcommand{\e}{\varepsilon}
\newcommand{\tto}{\rightrightarrows}
\newcommand{\gs}{\geqslant}
\newcommand{\ls}{\leqslant}
\renewcommand{\tilde}{\widetilde}
\renewcommand{\hat}{\widehat}
\newcommand{\norm}[1]{\left\| #1 \right\|}
\newcommand{\md}[3]{#1\equiv#2\;(\mathrm{mod}\;#3)}     
\newcommand{\gen}[1]{\left\langle #1 \right\rangle}
\renewcommand{\Re}{\operatorname{Re}}
\renewcommand{\Im}{\operatorname{Im}}
\newcommand{\zero}{\boldsymbol{0}}

\newcommand{\be}[1]{\textbf{\emph{#1}}}
\newcommand{\hhat}[1]{\hat{\! \hat{#1}}}

\newcommand{\fto}[1]{\xrightarrow{\hspace{4pt} #1 \hspace{4pt}}}
\newcommand{\flto}[1]{\xrightarrow{\quad #1 \quad}}



\newcommand{\dist}{\operatorname{dist}}
\newcommand{\esssup}{\operatorname{ess\:sup}}
\newcommand{\id}{\operatorname{id}}
\newcommand{\card}{\operatorname{card}}

\newcommand{\dmu}{\:\mathrm{d}\mu}
\newcommand{\dm}{\:\mathrm{d}m}
\newcommand{\dx}{\:\mathrm{d}x}
\newcommand{\dt}{\:\mathrm{d}t}
\newcommand{\dz}{\:\mathrm{d}z}
\newcommand{\dtheta}{\:\mathrm{d}\theta}
\newcommand{\dw}{\:\mathrm{d}w}

%Algebra
\newcommand{\Sym}{\operatorname {Sym}}
\newcommand{\Stab}{\operatorname {Stab}}
\newcommand{\M}{\operatorname{M}}
\newcommand{\GL}{\operatorname{GL}}
\newcommand{\PGL}{\operatorname{PGL}}
\newcommand{\SL}{\operatorname{SL}}
\newcommand{\PSL}{\operatorname{PSL}}
\newcommand{\Heis}{\operatorname{Heis}}
\newcommand{\Aff}{\operatorname{Aff}}
\newcommand{\Aut}{\operatorname{Aut}}
\newcommand{\image}{\operatorname{im}}
\newcommand{\Syl}[2]{\operatorname{\emph{Syl}}_{#1}\left(#2\right)}
\newcommand{\Hom}{\operatorname{Hom}}
\newcommand{\Tor}{\operatorname{Tor}}
\newcommand{\Gal}{\operatorname{Gal}}
\newcommand{\ch}{\operatorname{ch}}
\newcommand{\rad}{\operatorname{rad}}
\newcommand{\iso}{\cong}
\newcommand{\normal}{\unlhd}
\newcommand{\semi}{\rtimes}
\newcommand{\Nm}{\operatorname {N}}
\newcommand{\Tr}{\operatorname {Tr}}
\newcommand{\disc}{\operatorname {disc}}








%Euler Script Characters
\newcommand{\esa}{\EuScript{A}}
\newcommand{\esb}{\EuScript{B}}
\newcommand{\esc}{\EuScript{C}}
\newcommand{\esd}{\EuScript{D}}
\newcommand{\ese}{\EuScript{E}}
\newcommand{\esf}{\EuScript{F}}
\newcommand{\esg}{\EuScript{G}}
\newcommand{\esh}{\EuScript{H}}
\newcommand{\esi}{\EuScript{I}}
\newcommand{\esj}{\EuScript{J}}
\newcommand{\esk}{\EuScript{K}}
\newcommand{\esl}{\EuScript{L}}
\newcommand{\esm}{\EuScript{M}}
\newcommand{\esn}{\EuScript{N}}
\newcommand{\eso}{\EuScript{O}}
\newcommand{\esp}{\EuScript{P}}
\newcommand{\esq}{\EuScript{Q}}
\newcommand{\esr}{\EuScript{R}}
\newcommand{\ess}{\EuScript{S}}
\newcommand{\est}{\EuScript{T}}
\newcommand{\esu}{\EuScript{U}}
\newcommand{\esv}{\EuScript{V}}
\newcommand{\esw}{\EuScript{W}}
\newcommand{\esx}{\EuScript{X}}
\newcommand{\esy}{\EuScript{Y}}
\newcommand{\esz}{\EuScript{Z}}

%Calligraphic Characters
\newcommand{\cala}{\mathcal{A}}
\newcommand{\calb}{\mathcal{B}}
\newcommand{\calc}{\mathcal{C}}
\newcommand{\cald}{\mathcal{D}}
\newcommand{\cale}{\mathcal{E}}
\newcommand{\calf}{\mathcal{F}}
\newcommand{\calg}{\mathcal{G}}
\newcommand{\calh}{\mathcal{H}}
\newcommand{\cali}{\mathcal{I}}
\newcommand{\calj}{\mathcal{J}}
\newcommand{\calk}{\mathcal{K}}
\newcommand{\call}{\mathcal{L}}
\newcommand{\calm}{\mathcal{M}}
\newcommand{\caln}{\mathcal{N}}
\newcommand{\calo}{\mathcal{O}}
\newcommand{\calp}{\mathcal{P}}
\newcommand{\calq}{\mathcal{Q}}
\newcommand{\calr}{\mathcal{R}}
\newcommand{\cals}{\mathcal{S}}
\newcommand{\calt}{\mathcal{T}}
\newcommand{\calu}{\mathcal{U}}
\newcommand{\calv}{\mathcal{V}}
\newcommand{\calw}{\mathcal{W}}
\newcommand{\calx}{\mathcal{X}}
\newcommand{\caly}{\mathcal{Y}}
\newcommand{\calz}{\mathcal{Z}}

%Gothic Characters
\newcommand{\fraka}{\mathfrak{a}}
\newcommand{\frakb}{\mathfrak{b}}
\newcommand{\frakc}{\mathfrak{c}}
\newcommand{\frakd}{\mathfrak{d}}
\newcommand{\frake}{\mathfrak{e}}
\newcommand{\frakf}{\mathfrak{f}}
\newcommand{\frakg}{\mathfrak{g}}
\newcommand{\frakh}{\mathfrak{h}}
\newcommand{\fraki}{\mathfrak{i}}
\newcommand{\frakj}{\mathfrak{j}}
\newcommand{\frakk}{\mathfrak{k}}
\newcommand{\frakl}{\mathfrak{l}}
\newcommand{\frakm}{\mathfrak{m}}
\newcommand{\frakn}{\mathfrak{n}}
\newcommand{\frako}{\mathfrak{o}}
\newcommand{\frakp}{\mathfrak{p}}
\newcommand{\frakq}{\mathfrak{q}}
\newcommand{\frakr}{\mathfrak{r}}
\newcommand{\fraks}{\mathfrak{s}}
\newcommand{\frakt}{\mathfrak{t}}
\newcommand{\fraku}{\mathfrak{u}}
\newcommand{\frakv}{\mathfrak{v}}
\newcommand{\frakw}{\mathfrak{w}}
\newcommand{\frakx}{\mathfrak{x}}
\newcommand{\fraky}{\mathfrak{y}}
\newcommand{\frakz}{\mathfrak{z}}

\newcommand{\frakA}{\mathfrak{A}}
\newcommand{\frakB}{\mathfrak{B}}
\newcommand{\frakC}{\mathfrak{C}}
\newcommand{\frakD}{\mathfrak{D}}
\newcommand{\frakE}{\mathfrak{E}}
\newcommand{\frakF}{\mathfrak{F}}
\newcommand{\frakG}{\mathfrak{G}}
\newcommand{\frakH}{\mathfrak{H}}
\newcommand{\frakI}{\mathfrak{I}}
\newcommand{\frakJ}{\mathfrak{J}}
\newcommand{\frakK}{\mathfrak{K}}
\newcommand{\frakL}{\mathfrak{L}}
\newcommand{\frakM}{\mathfrak{M}}
\newcommand{\frakN}{\mathfrak{N}}
\newcommand{\frakO}{\mathfrak{O}}
\newcommand{\frakP}{\mathfrak{P}}
\newcommand{\frakQ}{\mathfrak{Q}}
\newcommand{\frakR}{\mathfrak{R}}
\newcommand{\frakS}{\mathfrak{S}}
\newcommand{\frakT}{\mathfrak{T}}
\newcommand{\frakU}{\mathfrak{U}}
\newcommand{\frakV}{\mathfrak{V}}
\newcommand{\frakW}{\mathfrak{W}}
\newcommand{\frakX}{\mathfrak{X}}
\newcommand{\frakY}{\mathfrak{Y}}
\newcommand{\frakZ}{\mathfrak{Z}}

%Lowercase Bold Letters
\newcommand{\bfa}{\mathbf{a}}
\newcommand{\bfb}{\mathbf{b}}
\newcommand{\bfc}{\mathbf{c}}
\newcommand{\bfd}{\mathbf{d}}
\newcommand{\bfe}{\mathbf{e}}
\newcommand{\bff}{\mathbf{f}}
\newcommand{\bfg}{\mathbf{g}}
\newcommand{\bfh}{\mathbf{h}}
\newcommand{\bfi}{\mathbf{i}}
\newcommand{\bfj}{\mathbf{j}}
\newcommand{\bfk}{\mathbf{k}}
\newcommand{\bfl}{\mathbf{l}}
\newcommand{\bfm}{\mathbf{m}}
\newcommand{\bfn}{\mathbf{n}}
\newcommand{\bfo}{\mathbf{o}}
\newcommand{\bfp}{\mathbf{p}}
\newcommand{\bfq}{\mathbf{q}}
\newcommand{\bfr}{\mathbf{r}}
\newcommand{\bfs}{\mathbf{s}}
\newcommand{\bft}{\mathbf{t}}
\newcommand{\bfu}{\mathbf{u}}
\newcommand{\bfv}{\mathbf{v}}
\newcommand{\bfw}{\mathbf{w}}
\newcommand{\bfx}{\mathbf{x}}
\newcommand{\bfy}{\mathbf{y}}
\newcommand{\bfz}{\mathbf{z}}




%Customized Theorem Environments
\newtheoremstyle%
{custom}%
{}%                         Space above
{}%													Space below
{}%													Body font
{}%                         Indent amount
{}%                         Theorem head font
{.}%                        Punctuation after heading
{ }%                        Space after heading
{\thmname{}%                Additional specifications for theorem head
\thmnumber{}%
\thmnote{\bfseries #3}}%

\newtheoremstyle%
{Theorem}%
{}%
{}%
{\itshape}%
{}%
{}%
{.}%
{ }%
{\thmname{\bfseries #1}%
\thmnumber{\;\bfseries #2}%
\thmnote{\;(\bfseries #3)}}%

%Theorem Environments
\theoremstyle{Theorem}
\newtheorem{theorem}{Theorem}[section]
\newtheorem{cor}{Corollary}[section]
\newtheorem{lemma}{Lemma}[section]
\newtheorem{prop}{Proposition}[section]
\newtheorem*{nonumthm}{Theorem}
\newtheorem*{nonumprop}{Proposition}
\theoremstyle{definition}
\newtheorem{definition}{Definition}[section]
\newtheorem*{answer}{Answer}
\newtheorem*{solution}{Solution}
\newtheorem*{nonumdfn}{Definition}
\newtheorem*{nonumex}{Example}
\newtheorem{ex}{Example}[section]
\theoremstyle{remark}
\newtheorem{remark}{Remark}[section]
\newtheorem*{note}{Note}
\newtheorem*{notation}{Notation}
\theoremstyle{custom}
\newtheorem*{cust}{Definition}
\fancypagestyle{firststyle}
{
  % \fancyhead[L]{\textbf{Name:}}
   \fancyhead[R]{\textbf{Worksheet 7: Comparison Tests, Ratio Test, Root Test, and Absolute Convergence}}
   \fancyfoot[R]{ Thomas Luckner } %{\footnotesize Page \thepage\ of \pageref{LastPage}}
}



\begin{document}
\thispagestyle{firststyle}
\pagestyle{plain}


\noindent Thoughts: \\
Comparison Tests- There are two of these! Say you have a series that looks vaguely like a geometric series, but is off by some little thing. This is why the comparison tests are nice! The key here is that you need to find another series that \textbf{LOOKS LIKE OR BEHAVES LIKE} the one you are questioned about that you know is convergent or divergent. Let's start with the basic one. 
\begin{theorem}[Comparison Test]
Suppose you have the series with sequence $a_n$ and another with sequence $b_n$ \textbf{both with positive terms}.
\begin{enumerate}[a.]
\item If the series for $b_n$ is convergent and $a_n\leq b_n$ for all $n$, then the series for $a_n$ is convergent.
\item If the series for $b_n$ is divergent and $a_n\geq b_n$ for all $n$, then the series for $a_n$ is divergent.
\end{enumerate}

\end{theorem}
\noindent You can show this easily with an example of some we already know. Consider $b_n=\dfrac{1}{2^n}$ and $a_n=\dfrac{1}{3^n}$ and assume we know the series for $b_n$ is geometric but not the series for $a_n$ (I know we do, but assume we do not for now). Since $a_n\leq b_n$ for all $n$ and the series for $b_n$ is convergent, so is the series for $a_n$! This is exactly as we would expect since the series for $a_n$ is a geometric series which is convergent!\\
I recommed trying the same thing for divergence and then trying with some $p$-series examples like the harmonic series.\\\\
Now there is another test which is more "all-ecompassing". I personally like this one since it does exactly what the comparison test does and more! 
\begin{theorem}[Limit Comparison Test]
Suppose sequences $a_n$ and $b_n$ of all positive terms. If 
\[
\lim_{n\rightarrow \infty}\dfrac{a_n}{b_n}=c
\]
where $c$ is a finite \textbf{POSITIVE} number, then either both series are convergent or both are divergent. 
\end{theorem}
\noindent Now I can show you how this works, but I think it is best if you try it with the previous example we had or use the following: $a_n=(n+1)/f(n)$ where $f(n)$ is a polynomial of degree $k$ and $b_n=(n+1)/g(n)$ where $g(n)$ is a poynomial of degree either $k+1$ or $k-1$. Try both of these cases and see what you get! They should yield different answers.
\newpage
Ratio Test- If you had a series about the sequence $a_n$ and $a_{n+1}/a_n$ goes to $L<1$ as $n$ goes to infinity, do you think the series for the sequence is convergent or divergent? 
The answer should be yes which is almost right. There is no other way to introduce this test without making an intuitive argument first like above! Let's make the statement now.
\begin{theorem}[Ratio Test]
The statement has three cases:
\begin{enumerate}[1.]
\item If $\ds \lim_{n\rightarrow \infty}\left|\dfrac{a_{n+1}}{a_n}\right| =L<1$, then the series for $a_n$ is \textbf{ABSOLUTELY} convergent. 
\item If the above $L>1$ or the limit is infinity, then the series is divergent.
\item If $L=1$ the test is inconclusive.
\end{enumerate}
\end{theorem}
Notice a couple things here. First our suspicion is almost true; we just needed absolute value. Second, This absolute convergence is not something that you have seen yet, but I will define it later. For now think of it as a stronger version of convergence. Third, you should be wondering why the case when $L=1$ is a thing! Let's go back to our intuition. If eventually it is 1, then they are equal eventually, but we have no information about the prior numbers so nothing can be said! There is a proof of this that is short and sweet, so please do ask if you'd like to see! Let's see a simple example. $a_n=\dfrac{n^3}{3^n}$. 
\[
\left|\dfrac{\dfrac{(n+1)^3}{3^{n+1}}}{\dfrac{n^3}{3^n}}\right|=\dfrac{3^n(n+1)^3}{3^{n+1}n^3}=\dfrac{(n+1)^3}{3n^3} \rightarrow \dfrac{1}{3}
\]
as $n$ goes to infinity. This is less than 1. Thus, the series for $a_n$ is absolutely convergent.\\
Root Test I do not want to waste time explaining this one since it has the same principle as the previous test! Here it is.
\begin{theorem}[Root Test]
The statement has three cases:
\begin{enumerate}[1.]
\item If $\ds \lim_{n\rightarrow \infty}\sqrt[n]{|a_n|}=L<1$, then the series for $a_n$ is \textbf{ABSOLUTELY} convergent. 
\item If the above $L>1$ or the limit is infinity, then the series is divergent.
\item If $L=1$ the test is inconclusive.
\end{enumerate}
\end{theorem}
Both of these tests are derived the same way. The use in this one comes in a little more specific situations. For example, you have a power of $n$ somewhere in your sequence for the series. Instead of trying to play with it, use the root test. Another common trick is to use this test when you want to get the limit of $\sqrt[n]{n}$. This limit is 1 and helps in some circumstances.
\newpage
\noindent Absolute Convergence- Here it is. This is a very very important concept to understand. \textbf{PLEASE READ THIS!} The first thing i want to say before I define thetype of convergence is that this is STRONGER than basic convergence. Thus, when this is not the case we have another name; Conditional Convergence. I will explain after.
\begin{definition}[Absolute Convergence]
Let $a_n$ be a sequence. If the series of $|a_n|$ is convergent, then the series of $a_n$ is absolutely convergent.
\end{definition}
This seems simple, but the question is why is this stronger? Well, If you have a sequence that is not all positive terms in the first place, the series of this sequence is less than the all positive series of it. Thus, if the positive sequence converges, so does the mixed series. Now, we have established that absolute convergence is stronger. Thus, the case where a series is convergent but not absolutely convergent needs a name. This is called Conditional Convergence. Examples of conditionally convergent series include alternating harmonic series, alternating p-series when $p<1$, etc. These counterexamples are something your instructor may ask you to know!
\newpage
Problems:
\begin{enumerate}[1.]
\item Use the comparison tests to determine if the series are convergent.
\begin{enumerate}[a.]
\item $\ds \sum_{n=1}^{\infty}\dfrac{9^n}{3+10^n}$
\item $\ds \sum_{n=1}^{\infty}\dfrac{1}{2n+3}$
\item $\ds \sum_{n=1}^{\infty} \dfrac{\sqrt{n^4+1}}{n^3+n^2}$
\item $\ds \sum_{n=1}^{\infty} \dfrac{1}{n!}$
\end{enumerate}
\item Determine if the series is absolutely convergent, conditionally convergent or divergent.
\begin{enumerate}[a.]
\item $\ds \sum_{n=1}^{\infty}\dfrac{n!}{100^n}$
\item $\ds \sum_{n=1}^{\infty} n\left(\dfrac{2}{3}\right)^n$
\item $\ds \sum_{n=1}^{\infty} \dfrac{(-2)^n}{n^n}$
\item $\ds \sum_{n=1}^{\infty} \dfrac{\cos(n\pi/3)}{n!}$
\item $\ds \sum_{n=1}^{\infty} \dfrac{(-1)^n}{n}$ (Alternating Harmonic Series)
\item $\ds \sum_{n=1}^{\infty} \dfrac{\cos(n)}{n^2}$

\end{enumerate}
\end{enumerate}
\end{document}







